\label{chap:6}

\subsection{Closing the Northern Boundary}

Motivated by the notions of far field forcing of the Southern Ocean, we investigated what considerations would be required to understand the dynamics of how this would be achieved. We found the boundary conditions chosen have large impacts of the eddy dynamics of a channel model along side complications in TEM theory to understand the surface layer make this a complex problem. Here we have hinted at some of the processes that may be involved in altering the SO ROC in different climatic states, but more fundamentally we have shown the limitations of channel models when investigating idealised SO ROC theory.

When the northern boundary is closed the SO ROC upper cell, which can be thought of as an extension of the AMOC, disappears. In all cases when the northern boundary is closed the traditional 3 cell SO ROC ceases to exist, depending on the boundary condition we may see a diabatically driven surface cell such as seen in our surface restoring runs (\fref{fig:L90rocyz}), but this is distinct from the globally connected SO ROC. When diabatic forcing becomes stronger in the sponge layer, the upper cell in the SO gains amplitude and becomes comparable to the observed SO ROC (and AMOC). In this regime the diapycnal eddy heat flux divergence is always of first order importance, counteracting the advective heat transport, leaving the surface heat forcing forcing as a smaller residual. 

The vertical integral of the diapycnal eddy heat flux convergence decreases by an order of magnitude for weaker diabatic forcing in the sponge layer, although the eddy fluxes themselves increase by a factor of 2 to 3 between a closed basin and a configuration with strong restoring to a prescribed temperature stratification in the northern sponge layer. Using fixed fluxes these changes are reconciled by the establishment of a strong internal boundary layer with a large vertical temperature gradient. 
As a result, both heat transport divergence by a weak ROC confined to strong northward deepening mixed layer, and the diapycnal eddy heat flux divergence appear as dipoles of opposing signs due to our choice TEM definitions. For the diapycnal eddy fluxes heat convergence in the upper half of the internal boundary layer and heat divergence in the lower half occurs, while the heat transport divergence associated with the ROC in the mixed layer shows the opposite.
Without diabatic forcing in a northern boundary sponge layer diabatic eddies cancel the effective surface buoyancy forcing, while the heat transport divergence by the ROC integrates to zero in the vertical. As a result, below the surface mixed layer the SO ROC completely collapses, because  the connection to an adiabatic pole-to-pole circulation ceases to exist. 

For fixed surface heat fluxes we have seen the SO ROC respond to changing northern boundary conditions via modulating the effective buoyancy forcing though diabatic eddies. Through out the thesis we have shown a selection of model runs performed particularly for the initial experiments varying the relaxation timescale. If we examine all the runs performed we can see a common factor between all these experiments, the northern boundary condition changes the background stratification. We can compare all these runs in terms of stratification by looking at Brunt V{\"a}is\"al\"a  frequency $N^2$. Here we plot the northern 100km integrated \% increase in $N^2$ vs \% decrease in domain integrated ROC strength for all experiments (varying stratification directly and indirectly though changing relaxation timescale). To see the relationship we linearise the curve $N^2$ vs domain integrated SO ROC by taking the natural logarithm of the increase in $N^2$ and perform a linear regression to give a line of best fit shown in figure \ref{fig:n2vsroc}. Here we see an exponential decay with $N^2$ \footnote{NB some models were never spun up and we do not have an even spread}. With fix surface fluxes it appears a very strong negative correlation with increased integrated $N^2$ and integrated SO ROC with R value of -0.95. 

%% 11
\begin{figure}[h]
\noindent \includegraphics[width=\textwidth]{../../Figures/ROCvn2.pdf}
\caption{Linear regression line of best fit for \% increase in integrated absolute ROC strength with natural logarithm of \% increase in $N^2$. y = mx +c , m = -27, c = -2, R value of -0.95 and a P value = 7 x $10^{-6}$.}
\label{fig:n2vsroc}
\end{figure}
%% 12

As we can see a clear trend of increasing integrated $N^2$ leading to decreasing integrated SO ROC and one of the major patterns we've seen with the collapsing SO ROC is increasing EKE that feeding into the increase of increased diabatic eddy heat fluxes. If we plot increase in domain integrated EKE that against increase in Northern Boundary integrated $N^2$ we see another strong correlation. Fig. \ref{fig:n2vseke} shows a 25 \% increase in $N^2$ gives rise to a dramatic 400\% increase in EKE. 

\begin{figure}[h]
\noindent \includegraphics[width=\textwidth]{../../Figures/EKEvn2.pdf}
\caption{Linear regression line of best fit for \% increase in EKE with \% increase in $N^2$. y = mx +c , m = 15, c = 15, R = 0.99, p value = 10$^{-8}$}
\label{fig:n2vseke}
\end{figure}

This implies that the choice of northern boundary condition thus the alteration of domain background stratification influences the domain wide dynamics e.g. lower relaxation timescales or deeper thermocline lead to increased amounts of EKE and greater eddy compensation before altering the surface forcing.  


\section{The Surface Heat Forcing}

Although we see that regardless of our choice in surface heat forcing we see a collapse in the SO ROC which can be weakly maintained even in absence of any surface heat forcing \fref{fig:noqrocyt} by a the diabatic forcing of the sponge layer along. We see large changes in the dynamics when the surface heat forcing is surface restoring, primarily in the surface mixed layer altering the diapycnal fluxes. When using a restoring boundary condition the main balance in the vertically integrated heat budget is somewhat different because no internal boundary layer develops and the northward deepening of the mixed layer is absent. The ROC is even more confined to the surface layer, and also in the case of a closed basin diabatic eddies and heat transport convergence by the ROC dominate the surface forcing. The role of the changing surface heat forcing mimics salinity driven (fixed fluxes) vs temperature driven (restoring) dynamics and this plays a role in setting the Southern Ocean dynamics in these differing regimes. For most of the worlds ocean buoyancy variations are dominated by temperature variations (variations in atmospheric temperature) most accurately represented by a restoring condition. The Southern Ocean, however, differs from this and is dominated by freshwater fluxes \citep{Karstensen2011} %Karstensen and Lorbacher 2011, A practical indicator for surface ocean heat and freshwater buoyancy fluxes and its application to theNCEPreanalysis data
which is more accurately represented by fixed-flux surface conditions\footnote{Very poorly constrained, so fixed-flux is simply more adequate \citep{Jansen2016}} although in reality the ocean is like likely to be best represented through a mixture of these boundary conditions \citep{Stewart2014}.
The choice of surface heat forcing alters some fundamental dynamics of the Southern Ocean from the depth of the mixed layer (forcing a shallowing), isopycnal slope, EKE and domain wide stratification. From our results we can see the choice of correctly matching the surface restoring profile and the Northern boundary condition has very large implications on the dynamics as well. Using surface restoring versus fix-fluxes has been to shown as large controll on degree of eddy compensation in ocean models \citep{Gent2015}. Due to these factors we can not find any such promising relation between the Northern Boundary stratification and SO ROC strength or EKE strength as we could see in the fixed flux cases. This should serve as strong consideration when setting up channel models to investigate SO theory. 

\section{Effects of Topography}

When considering the the Southern Ocean it is important to consider the effects of topography as it affects the ACC generating stationary meanders \cite{NaveiraGarabato2009} %RRS James Cook cruise 29, 01 Nov–22 Dec 2008. SOFine cruise report: Southern Ocean, National Oceanography Centre Southampton Cruise Report,
, alters the distribution of EKE \citep{Thompson2012}. We briefly ran some experiments to assess the possible implications of topography on the SO ROC response to altering the Northern boundary conditions. We see although the topography alters the distribution of EKE and alters some of the dynamics of the channel we still see the same 3 cell SO ROC collapsing when the northern boundary is closed only existing as an intense diabatically driven surface cell and like with the flat bottom fixed surface fluxes we see the generation of a deep mixed layer when the northern boundary is closed and a large increase in EKE associated with a diminished SO ROC.
%Does the sensitivity of Southern Ocean circulation depend upon bathymetric details?Andrew McC. Hogg1 and David R. Munday2 2017
\cite{Rosso2015} suggests that the role of topography may be further underestimated as a forward cascade of energy may lead to large influences on submesocale scales influencing diabatic eddies in the mixed layer.

%% Furture
\section{Summary and Future Work}

The original motivation for this work was to investigate far field forcing of the Southern ocean, we showed in all cases for our channel model the upper cell SO ROC collapses when diabatic forcing in the northern sponge-layer is absent. These results underscore the interhemispheric link between the SO ROC and the northern hemisphere AMOC. Such links were previously demonstrated in \citet{Gnanadesikan2000} and \citet{Wolfe2011}. More fundamentally however these results showed the limitations and considerations required when setting up a regional model of the Southern Ocean.
Here, by altering the northern boundary condition we showed how the SO ROC adjusts to changes in stratification at the northern end of the SO.
It should be stressed that the representation of the far-field forcing, i.e. NADW formation, by a sponge layer with a prescribed e-folding stratification is crude and should be tested against other ways of closing the SO ROC. Nevertheless, the absence of far-field forcing implies a disconnect between the SO ROC and the AMOC, and our results imply that in this case the upper cell of the SO ROC cannot be maintained. This result is independent of how the far field forcing is represented. What we do see is that the choice of northern boundary condition and surface forcing greatly alter the dynamics of the Southern Ocean from EKE to eddy saturation and ultimately how the collapse of the SO ROC is achieved is altered by the very sensitive set up of the channel model. 

The crucial role of diabatic eddy heat fluxes in the adjustment process to such changes highlights the need for a carefully designed diabatic eddy representation in the surface mixed layer of the ocean, which should also depend on the atmospheric state and forcing. Our results also imply that diabatic eddy fluxes can be essential in closing the heat budget and that coarse resolution ocean (and hence climate) models that do not represent mesoscale eddies without parameterizing the diabatic eddy fluxes cannot adequately close the heat budget and simulate the response to changing heat fluxes. We chose here to evaluate to the diabatic eddy heat fluxes as the diapycnal heat flux divergence in the heat budget to account for the variation between runs altering the effectiveness of various approximations in the mixed layer . The diapycnal eddy heat flux divergence strongly responds to changes in the northern boundary condition, becoming larger (increasing by 250\% in amplitude) when the stratification at the northern boundary is better able to freely evolve and is less constrained by the circulation and diabatic processes in the sponge layer. The resulting changes in diabatic forcing lead to a dramatic increase in surface mixed layer depth, which leads to enhanced baroclinic instability and larger EKE and EPE. This result is qualitatively robust to the surface boundary condition, but when surface restoring is applied instead of a flux formulation, a much smaller increase in diabatic fluxes occurs and the EKE is more constrained. Nevertheless, the role of diabatic eddy fluxes in the vertically integrated heat budget is equally large for both surface boundary conditions. It appears that the diabatic eddy heat flux divergence is sensitive to both the strength of the surface heat flux and the diabatic forcing in the sponge layer, but also to the surface boundary condition for temperature (buoyancy). 

Our channel model is very idealised, the next steps in understanding the role of diabatic eddies would involve the implementation of a more appropriate framework as well as perhaps moving to a less idealised set. The original theory we worked from uses buoyancy budgets not simply temperature alone to evaluate the SO ROC, in the cold temperatures of the Southern Ocean salinity plays are large role in setting the density of the ocean %CITE.
. Changing to a non linear Equation of state varying with Salinity and Temperature may lead to a more clear understanding of the role of surface forcing, with large discrepancies between fixed surface heat fluxes and surface restoring here suggesting differences between salinity driven and temperature driven changes, however the true surface forcing is likely to be a combinations. %cite
That leads us to the next improvement for future work, coupling to an atmosphere as these large changed in mixed layer heat flux divergence would have large implications for air-sea interactions that maybe better described by directly interacting with an atmosphere which would have a response to the changing conditions in the Southern Ocean creating feedbacks that may enhance the diabatic buoyancy forcing or dampen in. Alongside a coupled atmosphere we would also need to consider the role of sea ice on altering the surface buoyancy forcing with studies such as \cite{Ferreira2014}  % Antarctic sea ice control on ocean circulation in present and glacial climates.
suggesting that sea ice can play a large role on the forcing of the Deep Ocean. 

In chapter \ref{chap:5} we saw the effects of both the surface condition and the northern boundary condition had on eddy saturation suggesting closing the northern boundary reduced saturation as well as moving to a surface restoring boundary condition. The surface restoring is already known to have effects on setting the eddy saturation of a channel model, with stronger restoring giving stronger saturation \citep{Zhai2014}. Closing the northern boundary however increases eddy compensation and surface restoring reduces compensation. Mean models designed to investigate the impact of wind forcing may have been strongly influenced by the model set up leading to the disparity in findings over the years. 

In the introduction and in chapter \ref{chap:2} we introduced a the discussion of what parts of TEM theory from \citet{Marshall2003} and the original \citet{Andrews1976} is truly valid in the surface mixed which must be approached with approximations of varying limitations \citep{Plumb2005}. Part of the limitation is that we calculate the residual overturning stream function as an isopycnal streamfunction which deviates from the real residual streamfunction the most in the surface layers, this deviation is reduced by increasing the number of temperature layers the stream function is calculated over, however this is very computationally expensive. In atmospheric sciences \cite{Pauluis2011} developed a statistical transformed Eulerian mean (STEM) which when applied to the ocean does not break down at the surface \citep{wolfe2014}. STEM assumes that velocity and tracer fields can be represented by random processes possessing the same moments as the modelled field, this also makes for easier implementation which in light of the requirements for future development of TEM theory requiring more realistic set ups to fully understand the dynamics could be a tool to do so in a less computationally demanding manner.

There is still a lot more we need to understand about the Southern Ocean dynamics to fully comprehend all the feedbacks that may determine the SO ROC response to altered climates. The theory we have discussed here about the role diabatic eddies might play in that in turn high lighted the challenges in setting up experiments to test Southern Ocean theory as we demonstrate how the boundary conditions and forcing chosen may affect the SO ROC and eddy dynamics before even beginning to alter a chosen parameter. 

