

To give a full overview of the parameters used in the standard model set up below is an example data input file with the run time model flags and in italics a brief explanation. 

{\setlength{\parindent}{0cm}
\& \textbf{PARM01}

\textbf{tRef=Reference Profile,} \textit{This the reference temperature used in things like linear EOS, set as the sponge profile} 

\# viscosity 

\textbf{viscAh=12.000000,} \textit{Laplacian viscosity coefficient ($m^2/s$) Set so that horizontal Reynolds number is ~0, this scales with resolution of one less than the order (so linearly here) - basically it is to ensure that the velocity is not too fast that a wave can pass through a cell in less than one time step with the scaling relation  $\displaystyle A_h < \frac{L^2}{4 \Delta t}$ (Courant-Freidrichs-Lewy Constraint on viscosity). \citet{griffies2000} notes that it actually scales to one less power.}
 
\textbf{viscAz=0.000300,}\textit{vertical eddy coefficient. default}
 
\textbf{viscA4=9.0E8,}\textit{Bi-harmonic viscosity coefficient ($m^4/s$) With the relation $\displaystyle A_4 <= \frac{L^4}{32 \Delta t}$}. Allows a less viscous yet numerically stable simulation than harmonic (laplacian) as eddy viscosities can be focuses on the dynamics at the grid scale as large motions would be 'resolved'. 
 
 
\textbf{viscA4GridMax=0.500}\textit{Maximum grid dependent biharmonic viscosity }

 
\textbf{viscAhGridMax=0.500}\textit{Maximum lateral grid dependent eddy viscosity}
 
\# Diffusion 
 
\textbf{diffKhT=0.000000,}\textit{Laplacian diffusion of heat laterally ($m^2/s$) the boundary condition on this operator is $\displaystyle \frac{\partial}{\partial x}=\frac{\partial}{\partial y}=0$ on all boundaries. Default.}
 
\textbf{diffKzT=0.00000,} \textit{Laplacian diffusion of salt vertically can be set implicitly by setting implicitDiffusion to ’.TRUE.’ Default.}
 
\textbf{diffKhS=0.000000,}\textit{Laplacian diffusion of salt laterally ($m^2/s$) same applies as for T. Default. }
 
\textbf{diffKzS=0.00000,}\textit{Laplacian diffusion of salt vertically ($m^2/s$) same applies as for S.  Default.}

\# Advection Scheme

\textbf{tempAdvScheme=7,}\textit{The seventh-order one-step method with monotonicit-preserving limiter (OS7MP) \citep{Daru2004} and minimises numerical diffusion \citep{Ilıcak2012,shakespeare2012}}

\# Timestepping

\textbf{StaggerTimeStep=.TRUE.,}\textit{The stagger baroclinic time stepping rather than synchronous time stepping. The thermodynamics solver is delayed from half a time step, allowing the use of the most recent velocities to compute the advection terms}

\textbf{saltStepping=.FALSE.,}\textit{Salinity equation off.}

\# equation of state

\textbf{eosType='LINEAR'},\textit{Sets linear EOS, buoyancyRelation is automatically set to OCEANIC. For the nonlinear case, you need to generate a file of polynomial coeffcients called
POLY3.COEFFS. To do this, use the program utils/knudsen2/knudsen2.f instead of tAlpha and sBeta}

\textbf{tAlpha=2.0E-4,}\textit{thermal expansion coefficient in $K^{-1}$ for EOS. Default.}

\textbf{sBeta =0.,}\textit{salt is passive here (ppt$^{-1}$)}
 
\# Boundary conditions

\textbf{no\_ slip\_ sides=.TRUE.,} \textit{Free slip or no slip. Free slip give zero stress on boundaries (more convenient to code). No slip defines the normal gradient of a tangential flow such that the flow is zero on the boundary by adding an additional source term in cells next to the boundary}

\textbf{no\_ slip\_ bottom=.TRUE.,}

\textbf{bottomDragLinear=1.1E-03,}\textit{linear bottom-drag coefficient ( m/s )}

\textbf{ bottomDragQuadratic=0.000000E+00,}\textit{quadratic bottom-drag coeff. Default.}

\# physical parameters

\textbf{f0=-1.E-4,}\textit{Coriolis parameter ($s^{-1}$), $\displaystyle f=2\omega sin \phi$ Here this is set to the negative reference value and as beta is non zero f0 is the value of f at the southern edge of the domain.}

\textbf{beta=1E-11,}\textit{$\displaystyle \frac{\partial f}{\partial y}$ ($m^{-1}s^{-1}$)}

\textbf{gravity=9.81,}

\# exact volume conservation

\textbf{exactConserv=.TRUE.,}

\# C-V scheme for Coriolis term

\textbf{useCDscheme=.FALSE.,}

\# partial cells for smooth topography

\textbf{HfacMin=0.05,} \textit{ Limit partial cells to 50m thickness}

\# file I/O parameters

\textbf{readBinaryPrec=64,}\textit{double precision} 

\textbf{ useSingleCpuIO=.TRUE.,}

\textbf{ debugLevel=1,}

\# Default surface conditions

\textbf{rigidLid=.FALSE.,}

\textbf{implicitFreeSurface=.TRUE.,}\textit{these lines suppress the rigid lid formulation of the surface pressure inverter and activate the implicit
free surface form of the pressure inverter.}

\textbf{\&}

\textbf{\& PARM02}

\textbf{cg2dMaxIters=500,}\textit{Upper limit on 2d con. grad iterations, default is 150.}

\textbf{cg2dTargetResidual=1.E-9,} \textit{default is 1.E-7}

\textbf{cg3dMaxIters=40,}\textit{Sets the maximum number of iterations the three-dimensional, conjugate gradient solver}

\textbf{cg3dTargetResidual=1.E-9,}


\textbf{\&}


\textbf{\& PARM03}

\textbf{deltaT=450,}\textit{time steps scales with resolution}

\# 7.5 minute time step -> 69120 time steps/year

\textbf{nIter0=0,}\textit{Start at..}

\textbf{nTimeSteps=1382400,}\textit{20 years}

\textbf{cAdjFreq = -1.,}\textit{Convective adjustment interval (s), the frequency at which the adjustment algorithm is called to a non-zero value, a negative value sets to tracer time step}

\textbf{abEps=0.1,}\textit{ Adams-Bashforth stabilizing parameter, required as staggerTimeStep is set to TRUE}

\# Permanent restart/checkpoint file interval ( s )

\# Try 10 year interval (two per run)

\textbf{pChkptFreq=311040000.00,}\textit{Checkpoint file interval (s)}

\textbf{dumpFreq=0.0,} \textit{Model state write out interval (s)}
 
\textbf{monitorFreq=0.0,} \textit{Monitor output interval (s)}

\textbf{dumpInitAndLast=.TRUE.,}

\textbf{pickupStrictlyMatch=.FALSE.,}

\textbf{\&}

\textbf{\& PARM04}

\textbf{usingCartesianGrid=.TRUE.,}\textit{Uses cartesian co-ordinate system with uniform grid space via dXspacing or dYspacing or by vectors DelX or DelY in (m)}

\textbf{usingSphericalPolarGrid=.FALSE.,}

\textbf{delXfile='delX',}\textit{Can be set as a 1D array file or as gridlines * spacing i.e 300*5E3 300 grid lines at 5km spacing}

\textbf{delYfile='delY',}

\textbf{delZfile='delZ'}

\textbf{\&}

\textbf{\& PARM05}

\textbf{surfQfile='Qsurface'}

\textbf{bathyFile='topog'}

\textbf{zonalWindFile='Wind',}

\textbf{meridWindFile=,}

\textbf{hydrogThetaFile='T.init',}

\textbf{\&}
}