\documentclass[a4paper,12pt, openright, titlepage]{book}
\usepackage[toc,page]{appendix}
\usepackage{tocbibind}
\usepackage{tocloft}
\usepackage{lipsum}
\usepackage{graphicx}
\usepackage{fancyhdr}
\usepackage{natbib}
\usepackage{float}
\usepackage{pdfpages}
\usepackage{gensymb}
\usepackage{amsmath}
\usepackage{mathtools}
\usepackage{wrapfig}
\usepackage{tabularx}
\usepackage{sidecap}
\usepackage{amssymb}
\usepackage{textcomp}
\usepackage{wasysym}
\usepackage{rotating}
\usepackage[hidelinks]{hyperref}
\usepackage{color}
\usepackage{float}
\usepackage{mathtools}
\usepackage{xfrac}
\usepackage{setspace}
\usepackage{caption}
\usepackage{lineno}
\usepackage[plain]{fancyref}
\renewcommand{\frefeqname}{Eq.}
\renewcommand{\freffigname}{Fig.}
%% Only put on page xx if fig or eq is far away!!
\renewcommand*{\reftextfaceafter}{\unskip}
\renewcommand*{\reftextafter}{\unskip}
\renewcommand*{\reftextfacebefore}{\unskip}
\renewcommand*{\reftextbefore}{\unskip}
\makeatletter
\let\saved@reftextfaraway\reftextfaraway
\renewcommand*{\reftextfaraway}[1]{%
  \begingroup
    \def\ref@unknown@value{??}%
    \ifx\@tempa\ref@unknown@value
      \count@=0 %
    \else
      \count@\thevpagerefnum\relax
      \advance\count@ by -\@tempa\relax
      \ifnum\count@<0 \count@=-\count@\fi
    \fi
    \ifnum\count@<5 %
      \unskip
    \else
      \saved@reftextfaraway{#1}%
    \fi
  \endgroup
}
\makeatother
\usepackage[lofdepth,lotdepth]{subfig}
\usepackage[normalem]{ulem}
\usepackage[makeroom]{cancel}
\usepackage{multirow}
\usepackage{hyperref}
\usepackage{cleveref} % must be last referencing package loaded
\usepackage[normalem]{ulem}
\usepackage{mathtools}
\usepackage{pdflscape}
\usepackage[british]{babel}
\usepackage{enumitem}
\usepackage{pbox}
\usepackage{siunitx}

\newcommand\finline[3][]{\begin{myfont}[#1]{#2}#3\end{myfont}}%
\newlist{abbrv}{itemize}{1}
\setlist[abbrv,1]{label=,labelwidth=1in,align=parleft,itemsep=-0.5\baselineskip,leftmargin=1in}
\usepackage[a4paper, top=80pt, bottom=60pt, left=60pt, right=60pt]{geometry}
\bibpunct{(}{)}{;}{a}{,}{,}
\pagestyle{fancy}
\linespread{1.25}
\setlength{\headheight}{15pt}
\setlength{\oddsidemargin}{15mm}
\raggedright
\setlength{\parskip}{10pt }
\setlength{\parindent}{0.5 cm}
\setlength{\textwidth}{148 mm}
\fancyheadoffset{1 cm}
\fancyhead[RO]{Helen Burns}
\fancyhf{}
\fancyhead[R]{Helen Burns}
\fancyhead[L]{\leftmark} % 1. sectionname
\fancyfoot[C]{\thepage}
% Add a roman numeral in title text
\makeatletter
\newcommand{\rmnum}[1]{\romannumeral #1}
\newcommand{\Rmnum}[1]{\expandafter\@slowromancap\romannumeral #1@}
\makeatother
%%%%%%%%%%%%%%%%%%%%%
%%%% To add code %%%%
\usepackage{listings}
\usepackage{color}

\definecolor{dkgreen}{rgb}{0,0.6,0}

\definecolor{gray}{rgb}{0.5,0.5,0.5}

\definecolor{mauve}{rgb}{0.58,0,0.82}

\lstset{frame=tb,
  language=Fortran,
  aboveskip=2mm,
  belowskip=2mm,
  showstringspaces=false,
  columns=flexible,
  basicstyle={\small\ttfamily},
  numbers=none,
  numberstyle=\tiny\color{gray},
  keywordstyle=\color{blue},
  commentstyle=\color{dkgreen},
  stringstyle=\color{mauve},
  breaklines=true,
  breakatwhitespace=true
  tabsize=3
}
%\usepackage[toc,page]{appendix}
\usepackage[nonumberlist,toc,acronym,automake]{glossaries}
\newglossary[slg]{symbolslist}{syi}{syg}{Symbolslist}
\glsaddkey{unit}{\glsentrytext{\glslabel}}{\glsentryunit}{\GLsentryunit}{\glsunit}{\Glsunit}{\GLSunit}

\makeglossaries
%%%%%%%%%%%%%%%%%%%%%%%%%%%%%%%%%%%%%%%%%%%%%%%%%%%%%%%%%%%%%%%%%
%%%%                                                         %%%%
%%%%------------------------Glossary-------------------------%%%%
%%%%                                                         %%%%
%%%%%%%%%%%%%%%%%%%%%%%%%%%%%%%%%%%%%%%%%%%%%%%%%%%%%%%%%%%%%%%%%
\newglossarystyle{symbunitlong}{%
\setglossarystyle{long3col}% base this style on the list style
\renewenvironment{theglossary}{% Change the table type --> 3 columns
  \begin{longtable}{lp{0.6\glsdescwidth}>{\centering\arraybackslash}p{2cm}}}%
  {\end{longtable}}%
%
\renewcommand*{\glossaryheader}{%  Change the table header
  \bfseries Symbol & \bfseries Description & \bfseries Unit \\
  \hline
  \endhead}
\renewcommand*{\glossentry}[2]{%  Change the displayed items
\glstarget{##1}{\glossentryname{##1}} %
& \glossentrydesc{##1}% Description
& \glsunit{##1}  \tabularnewline
}
}
% ======================= SYMBOLS LIST ======================== %
\newglossaryentry{symb:a}{name=\ensuremath{_a},
   description={denotes ageostophic},
   unit={},
   type=symbolslist}
\newglossaryentry{symb:b}{name=\ensuremath{b},
   description={buoyancy},
   unit={},
   type=symbolslist}
\newglossaryentry{symb:be}{name=\ensuremath{\beta},
   description={Coriolis ratio},
   unit={},
   type=symbolslist}
\newglossaryentry{symb:bo}{name=\ensuremath{\mathcal{B}},
   description={buoyancy forcing},
   unit={},
   type=symbolslist}
\newglossaryentry{symb:chi}{name=\ensuremath{\chi},
   description={},
   unit={},
   type=symbolslist}
\newglossaryentry{symbLCp}{name=\ensuremath{C{_p}},
description={ Specific heat capacity. (Sea water = 3985 JKg$^{-1}$K$^{-1}$}),
unit={JKg$^{-1}$K$^{-1}$},
   type=symbolslist}
\newglossaryentry{symb:D}{name=\ensuremath{D},
   description={Material derivative},
   unit={},
   type=symbolslist}
\newglossaryentry{symb:Ek}{name=\ensuremath{Ek},
   description={Ekman number},
   unit={},
   type=symbolslist}
\newglossaryentry{symb:eta}{name=\ensuremath{\eta},
   description={Bottom topography},
   unit={m},
   type=symbolslist}
\newglossaryentry{symb:etab}{name=\ensuremath{\eta _b},
   description={Sea Surface Height},
   unit={m},
   type=symbolslist}
\newglossaryentry{symb:f}{name=\ensuremath{f},
   description={coriolis frequency},
   unit={},
   type=symbolslist}
\newglossaryentry{symb:Fu}{name=\ensuremath{F},
   description={Function (variables)},
   unit={},
   type=symbolslist}
\newglossaryentry{symb:G}{name=\ensuremath{\mathcal{G}},
   description={Green's function ?},
   unit={},
   type=symbolslist}
\newglossaryentry{symb:hml}{name=\ensuremath{hml},
   description={Mixed layer Depth},
   unit={m},
   type=symbolslist}
\newglossaryentry{symb:J}{name=\ensuremath{J},
   description={Jacobian},
   unit={},
   type=symbolslist}
\newglossaryentry{symb:K}{name=\ensuremath{K},
   description={Eddy transfer coefficient},
   unit={},
   type=symbolslist}
\newglossaryentry{symb:Ka}{name=\ensuremath{\kappa},
   description={Eddy diffusivity constant},
   unit={},
   type=symbolslist}
\newglossaryentry{symb:L}{name=\ensuremath{L},
   description={Length},
   unit={m},
   type=symbolslist}
\newglossaryentry{symb:N2}{name=\ensuremath{N^2},
   description={Brunt V{\"a}is\"al\"a  frequency},
   unit={},
   type=symbolslist}
\newglossaryentry{symb:omega}{name=\ensuremath{\Omega },
   description={},
   unit={},
   type=symbolslist}
\newglossaryentry{symb:phi}{name=\ensuremath{\Phi },
   description={},
   unit={},
   type=symbolslist}
\newglossaryentry{symb:psi}{name=\ensuremath{\Psi },
   description={Stream function},
   unit={m$^3$ /s},
   type=symbolslist}
\newglossaryentry{symb:Q}{name=\ensuremath{Q},
   description={Heat Flux},
   unit={w/m$^2$},
   type=symbolslist}
\newglossaryentry{symb:rho}{name=\ensuremath{\rho },
   description={Density, $_0$ denotes reference density.},
   unit={kg/m$^2$},
   type=symbolslist}
\newglossaryentry{symb:ro}{name=\ensuremath{Ro },
   description={Rossby number},
   unit={},
   type=symbolslist}
\newglossaryentry{symb:sp}{name=\ensuremath{S_p },
   description={Isopycnal slope},
   unit={},
   type=symbolslist}
\newglossaryentry{symb:Tau}{name=\ensuremath{\tau _s},
   description={Wind Stress},
   unit={},
   type=symbolslist}
\newglossaryentry{symb:TauCurl}{name=\ensuremath{\nabla \times \tau _s},
   description={Wind Stress Curl},
   unit={},
   type=symbolslist}
\newglossaryentry{symb:theta}{name=\ensuremath{\theta },
   description={Potential temperature},
   unit={$^o$ C},
   type=symbolslist}
\newglossaryentry{symb:U}{name=\textbf{u},
   description={Velocity vector (x,y,z,t), $u$ = zonal, $v$ = meridional, $w$ = vertical},
   unit={m/s},
   type=symbolslist}
\newglossaryentry{symb:Pi}{name=\ensuremath{\pi},
   description={Geometrical value},
   unit={},
   type=symbolslist}
\newglossaryentry{P}{name=\ensuremath{P_b},
        description={Bottom Pressure},
        unit={N/m$^2$},
        type=symbolslist}
\newglossaryentry{vor}{name=\ensuremath{q},
        description={vorticity},
        unit={},
        type=symbolslist}
% ======================= ACRONYMS LIST ======================= %
\newacronym[sort=AABW]{AABW}{AABW}{Antarctic Bottom Water}
\newacronym[sort=AAIW]{AAIW}{AAIW}{Antarctic Intermediate Water}
\newacronym[sort=ACC]{ACC}{ACC}{Antarctic Circumpolar Current}
\newacronym[sort=AMOC]{AMOC}{AMOC}{Atlantic Meridional Overturning Circulation}
\newacronym[sort=AOGCM]{AOGCM}{AOGCM}{Atmosphere-Ocean Coupled Global Climate Model} 
\newacronym[sort=CDW]{CDW}{CDW}{Circumpolar Deep Water}
\newacronym[sort=CFL]{CFL}{CFL}{Courant–Friedrichs–Lewy}
\newacronym[sort=Fortran]{Fortran}{Fortran}{Formula Translating System}
\newacronym[sort=EKE]{EKE}{EKE}{Eddy Kinetic Energy}
\newacronym[sort=EPE]{EPE}{EKE}{Eddy Potential Energy}
\newacronym[sort=GCM]{GCM}{GCM}{General Circulation Model}
\newacronym[sort=GPE]{GPE}{GPE}{Gravitational Potential Energy}
\newacronym[sort=GS]{GS}{GS}{Gulf Stream}
\newacronym[sort=KE]{KE}{KE}{Kinetic Energy}
\newacronym[sort=KPP]{KPP}{KPP}{K-profile parameterization}
\newacronym[sort=LCDW]{LCDW}{LCDW}{Lower Circumpolar Deep Water}
\newacronym[sort=LHS]{LHS}{LHS}{Left Hand Side}
\newacronym[sort=ITCZ]{ITCZ}{ITCZ}{Inter Tropical Convergence Zone}
\newacronym[sort=MITgcm]{MITgcm}{MITgcm}{Massachusetts Institute of Technology General Circulation Model}
\newacronym[sort=MOC]{MOC}{MOC}{Meridional Overturning Circulation}
\newacronym[sort=NADW]{NADW}{NADW}{North Atlantic Deep Water}
\newacronym[sort=NCEPNCAR]{NCEPNCAR}{NCEPNCAR}{National Centers for Environmental Prediction-National Centre for Atmospheric Research}
\newacronym[sort=NERC]{NERC}{NERC}{Natural Environment Research Council}
\newacronym[sort=netCDF]{netCDF}{netCDF}{Network Common Data Form}
\newacronym[sort=NOC]{NOC}{NOC}{National Oceanography Centre}
\newacronym[sort=PV]{PV}{PV}{Potential Vorticity}
\newacronym[sort=PF]{PF}{PF}{Polar Front}
\newacronym[sort=POP]{POP}{POP}{ Parallel Ocean Programme}
\newacronym[sort=RHS]{RHS}{RHS}{Right Hand Side}
\newacronym[sort=RMOC]{RMOC}{RMOC}{Residual Meridional Overturning Circulation}
\newacronym[sort=ROC]{ROC}{ROC}{Residual Overturning Circulation}
\newacronym[sort=QG]{QG}{QG}{Quasi-Geostophic}
\newacronym[sort=SAF]{SAF}{SAF}{Southern Antactarctic Front}
\newacronym[sort=SAM]{SAM}{SAM}{Southern Annular Mode}
\newacronym[sort=SAMW]{SAMW}{SAMW}{Sub Antarctic Mode Water}
\newacronym[sort=SOMOC]{SOMOC}{SOMOC}{Southern Ocean Meridional Overturning Circulation}
\newacronym[sort=SOROC]{SOROC}{SOROC}{Southern Ocean Residual Overturning Circulation}
\newacronym[sort=SSH]{SSH}{SSH}{Sea Surface Height}
\newacronym[sort=SSHA]{SSHA}{SSHA}{Sea Surface Height Anomaly}
\newacronym[sort=SST]{SST}{SST}{Sea Surface Temperature}
\newacronym[sort=SSTA]{SSTA}{SSTA}{Sea Surface Temperature Anomaly}
\newacronym[sort=SSSA]{SSSA}{SSSA}{Sea Surface Salinity Anomaly}
\newacronym[sort=STF]{STF}{STF}{Subtropical Front}
\newacronym[sort=TIW]{TIW}{TIW}{Tropical Instability Wave}
\newacronym[sort=TEM]{TEM}{TEM}{Transformed Eulerian Mean}
\newacronym[sort=TKE]{TKE}{TKE}{Turbulent Kinetic Energy}
\newacronym[sort=TNA]{TNA}{TNA}{Tropical North Atlantic}
\newacronym[sort=U]{U}{U}{Zonal Velocity in m/s}
\newacronym[sort=UCDW]{UCDW}{UCDW}{Upper Circumpolar Deep Water}
\newacronym[sort=V]{V}{V}{Meridional Velocity in m/s}

%-----------------------------------------------------------------------------
% ======================= GLOSSARY LIST ======================== %
\newglossaryentry{adiabatic}
{
name={adiabatic},
description={},
sort=adiabatic
}
%-------------------------------
\newglossaryentry{baroclinic}
{
name={baroclinic},
description={},
sort=baroclinic
}
%-------------------------------
\newglossaryentry{barotropic}
{
name={Barotropic},
description={},
sort=barotropic
}
%-------------------------------
\newglossaryentry{Brunt Vaisala frequency}
{
name={Brunt V{\"a}is\"al\"a  frequency},
description={Largest temporal frequency that internal waves can have},
sort=Brunt Vaisala frequency
}

%-------------------------------
\newglossaryentry{diabatic}
{
name={Diabatic},
description={},
sort=diabatic
}
%-------------------------------
\newglossaryentry{e-folding}
{
name={e-folding},
description={A factor of 1/$e$ , space scale for change of 1/$e$ if exponential decay is assumed},
sort=e-fold
}
%-------------------------------
\newglossaryentry{Geostrophic}
{
name={geostrophic},
description={},
sort=geostrophic,
type = main
}
%-------------------------------
\newglossaryentry{isotherm}
{
name={Isotherm},
description={Surface of equal temperature},
sort=isotherm
}
%-------------------------------
\newglossaryentry{incompressibility}
{
name={Incompressibility},
description={incompressibility},
sort=incompressibility
}
%-------------------------------
\newglossaryentry{psi}
{
name={$\ensuremath{\psi}$},
description={ Stream function (m$^{2}$s$^{-1}$)},
sort=psi
}
%-------------------------------
\newglossaryentry{psir}
{
name={$\ensuremath{\psi_{res}}$},
description={ Residual overturning stream function (m$^{2}$s$^{-1}$)},
sort=psir
}
%-------------------------------
 \newglossaryentry{psu}
{
  name={psu},
  description={ The Practical Salinity Scale defines salinity in terms of the conductivity ratio of a sample to that of a solution of 32.4356 g of KCl at $15^{o}$C in a 1 kg solution. A sample of seawater at 15$^{o}$C with a conductivity equal to this KCl solution has a salinity of exactly 35 practical salinity units (psu)           },
  sort=psu
}
%-------------------------------
\newglossaryentry{rho}
{
name={$\ensuremath{\rho}$},
description={ Density of sea water (1030 kg m$^{-3}$)},
sort=rho
}
%-------------------------------
\newglossaryentry{Rsbrad}
{
name={Rossby radius},
description={The scale for mesoscale eddies to form baroclinic instability. $\displaystyle {L_d = \frac{NH}{f}}$, where N is the bouyancy frequency, H is thickness and f is the coriolis frequency.  },
sort=rossby
}
%-------------------------------
\newglossaryentry{SO}
{
  name={Southern Ocean},
  description={Region of ocean around Antarctica, although not defined as a subregion of the world ocean by the International Hydrographic Organization, it represents a region of differing dynamics separate from the Atlantic, Pacific and Indian oceans \citep{tomczak2003}.},
  sort=So
}
%-------------------------------
\newglossaryentry{Sv}
{
  name={Sverdrups},
  description={ 10$^{6}$ m/s$^{2}$                 },
  sort=Sv
}
%-------------------------------

%-------------------------------
%-------------------------------
%-------------------------------
%-------------------------------

%









% ENVIRONMENT FOR NEW FONT
\newenvironment{myfont}[2][]{\csname#2\endcsname[#1]}{}

\newcommand{\Alpine}[1][]{\fontfamily{Alpine}#1\selectfont}

\AtBeginDocument{ 
%%%%%% So I made my own titlepage %%%%%
\begin{titlepage}

\newcommand{\HRule}{\rule{\linewidth}{0.5mm}} % Defines a new command for the horizontal lines, change thickness here

\center % Center everything on the page
 
%----------------------------------------------------------------------------------------
%   HEADING SECTIONS
%----------------------------------------------------------------------------------------

\textsc{\LARGE University of Southampton}\\[1.3cm] % Name of your university/college
\textsc{\Large National Oceanography Centre}\\[0.4cm] % Major heading such as course name


%----------------------------------------------------------------------------------------
%   TITLE SECTION
%----------------------------------------------------------------------------------------

\HRule \\[0.55cm]
{ \Huge \bfseries Diabatic eddies in idealised channel models}\\[0.35cm] % Title of your document
\HRule \\[1.1cm]
 %----------------------------------------------------------------------------------------
\textsc{Thesis for the degree of Doctor of Philosophy}\\[1.2cm]
%----------------------------------------------------------------------------------------
%   DATE SECTION
%----------------------------------------------------------------------------------------
%----------------------------------------------------------------------------------------
%   AUTHOR SECTION
%----------------------------------------------------------------------------------------

\begin{minipage}{0.4\textwidth}
\begin{flushleft} \large
\emph{Author:}\\
 Helen Burns
\end{flushleft}
\end{minipage}
~
\begin{minipage}{0.4\textwidth}
\begin{flushright} \large
\emph{Supervisors:} \\
Sybren Drijfhout and Alberto Naveira Garabato % Supervisor's Name
\end{flushright}
\end{minipage}\\[2cm]

%----------------------------------------------------------------------------------------
%   DATE SECTION
%----------------------------------------------------------------------------------------

{\large \today}\\[2cm] % Date, change the \today to a set date if you want to be precise

%----------------------------------------------------------------------------------------
%   LOGO SECTION
%----------------------------------------------------------------------------------------
%\includegraphics[width=0.8\textwidth]{logos.png}%

\vfill %Fill the rest of the page with whitespace

\end{titlepage}

\justify
%\begin{abstract}
%\newcommand{\HRule}{\rule{\linewidth}{0.5mm}}
\center % Center everything on the page
}


% Centered title for ToC, LoF, LoT
\renewcommand{\cfttoctitlefont}{\hfill\Huge\bfseries}
\renewcommand{\cftaftertoctitle}{\hfill}
\renewcommand{\cftloftitlefont}{\hfill\Huge\bfseries}
\renewcommand{\cftafterloftitle}{\hfill}
\renewcommand{\cftlottitlefont}{\hfill\Huge\bfseries}
\renewcommand{\cftafterlottitle}{\hfill}

% Leaders for chapter entries
\renewcommand\cftchapdotsep{\cftdotsep}

% Add space to account for new chapter numbering schema
\renewcommand\cftchapnumwidth{3em}
\renewcommand\cftsecindent{3em}

% Redefine representation for chapter (and section) counters
\renewcommand\thechapter{\arabic{chapter}}
\renewcommand\thesection{\arabic{chapter}.\arabic{section}}



\begin{document}

%------------------------------------------------------
\thispagestyle{empty}
\frontmatter
\textsc{\large \uline{Abstract}}\\[0.1cm] 
\addcontentsline{toc}{chapter}{Abstract}

\textsc{ \textbf{\textsc{UNIVERSITY OF SOUTHAMPTON}}}\\ 
%

\textsc{\small FACULTY OF NATURAL AND ENVIRONMENTAL SCIENCES}\\[0.05cm] 
\textsc{\footnotesize National Oceanography Centre }\\[0.05cm]
\textsc{\uline{Doctor of Philosophy}}\\[0.1cm]

\textbf{\textsc{\large Diabatic eddies in idealised channel models}}\\[0.01cm]
{by Helen Burns}\\[0.2cm]
\justify

%Motivation
\linespread{1}
%NEEDS UPDATING
%In this thesis we investigate the role of diabatic eddies in the southern ocean. To do this we used heat budgets in a \gls{TEM} framework to establish the importance of diabatic eddies in moderating the effective heat forcing of the southern ocean. This is a theoretical concept that first must be tested in an idealised channel model. Here we outline the formation of a model to investigate this concept, starting with the rational that a closed northern boundary should lead to a collapse in the \gls{SOROC}, which is demonstrated in our first set of experiments. To understand fully the dynamics of the SO ROC we investigate the effects of altering the northern boundary condition as well as the effects of the surface heat forcing on this response. We find that altering the surface forcing from fixed fluxes to surface restoring alters the dynamics of the mixed layer and eddy compensation balance of the SO ROC. Although we still see the SO ROC collapse in both simulations with surface restoring and fixed surface fluxes the dynamics are altered. This theme is continued when looking at the ACC which suggests that the surface boundary condition also affects the degree of eddy saturation. These results suggest that there are feedbacks that must be considered when setting up channel models as the surface boundary condition or the northern boundary condition may decided the results of other perturbation experiments such as altering windstress before the alteration of that forcing. Diabatic eddies could play an important role in setting the response of the SO ROC to future climate but the mechanisms are very sensitive to model boundary conditions that must be considered when setting up ocean models to investigate the fundamental theory of the complex SO ROC. 

%------------------------------------------------------



%\setcounter{page}{2}
\newpage{}
\textsc{\large \uline{Acknowledgements}}\\
\addcontentsline{toc}{chapter}{Acknowledgements}
\clearpage

\chapter[Declaration of authorship]{}
%\addcontentsline{toc}{chapter}{Declaration of authorship}
%--------------------------------------------------------------------------------------------
%\thispagestyle{empty}
. \\[-6.51cm]
\textbf{\textsc{\Large Academic Thesis: Declaration Of Authorship}}\\[0.5cm]


\noindent I, Helen Burns, declare that this thesis and the work presented in it are my own and has been generated by me as the result of my own original research.\\[0.3cm]

%\textbf{\textsc{\large Title}}\\[0.3cm]

\noindent I confirm that:\\[-1.5cm]
\begin{enumerate}
\item This work was done wholly or mainly while in candidature for a research degree at this University;
\item Where any part of this thesis has previously been submitted for a degree or any other qualification at this University or any other institution, this has been clearly stated;
\item Where I have consulted the published work of others, this is always clearly attributed;
\item Where I have quoted from the work of others, the source is always given. With the exception of such quotations, this thesis is entirely my own work;
\item I have acknowledged all main sources of help;
\item Where the thesis is based on work done by myself jointly with others, I have made clear exactly what was done by others and what I have contributed myself;
\item Either none of this work has been published before submission, or parts of this work have been published as: \\

\end{enumerate}

Signed:\hrulefill\\%[1.5cm]


Date:\hrulefill\\
%------------------------------------------------------
\clearpage
\tableofcontents
\clearpage
\listoftables
\listoffigures

\clearpage
\pagenumbering{arabic}
\setcounter{page}{1}
\mainmatter
% Chapter 1
%Introduction dynamics and TEM theory
\chapter[Introduction]{Introduction}

%%%%%%%%%%%%%%%%%%%%%%%%%%%%%%%%%%%%
%\textcolor{red}{\textbf{To be covered:}}
%\begin{itemize}
%\item[]\textcolor{red}{Introduction to Southern Ocean Dynamics, its role in global ocean and climate, complexities with eddies and mean flow, further complexities...}
%\item[]\textcolor{red}{Residual mean theory and TEM theory, the  Assumptions made, MR03 Ferrari and Plumb etc}
%\item[]\textcolor{red}{Outline the the hypothesis here, the motivation and the main questions to focus on}
%\item[]\textcolor{red}{Out line the model, introduce the MITgcm and how to set up the model for this purpose}
%\end{itemize}
%%%%%%%%%%%%%%%%%%%%%%%%%%%%%%%%%%%%
\section*{Overview}

The \gls{SO} is unique, containing a band of latitudes of no continuous
topography above 1800m below sea level \citep{hallberg2006}. Strong
polar Westerlies allow for a strong zonal \gls{ACC} that flows Eastwards around the continent between 56$^o$ \& 62$^o$ S \citep{johnson1989}. This lack of topography and zonal flow results in unique dynamics that play and important role in connecting all the ocean basins as well as completing the global \gls{MOC} \citep{marshall2012} as seen in \fref{fig:MS2012}. The Southern Ocean is an important sink of $CO_2$ with the \gls{ACC} accounting for around 40$\%$ of the global uptake \citep{Mignone2006}. 

\begin{figure}[h]
\centering
\includegraphics[width=0.8\textwidth]{../../Figures/MOC2012.png}
\caption{Southern Ocean's role in the global overturning schematic from \cite{marshall2012}, with blues and purples representing denser water masses and reds and yellows less dense water masses.}
\label{fig:MS2012}
\end{figure}

Changes in the Southern Ocean overturning would alter the Southern Ocean carbon sink changing the upwelling of deep waters rich in Dissolved Inorganic Carbon (DIC), and altering the partial pressure of CO$_2$ in the ocean. Between 1981 and 2004 the Southern Ocean sink has decreased by 0.08 petagrams of C per year due to increased wind stress showing that the CO$_2$ sink cannot be assumed to stay constant\citep{lequere2007}. 

However the dynamics of the Southern Ocean are complex and poorly understood, with sparse observations in time and space, it is therefore of great interest to fully understand the dynamics of the Southern Ocean and the response to future climate change. Of particular interest in this thesis is how changes in the northern Hemisphere might alter the Southern Ocean. Inter-hemispheric links between the SO ROC and Northern Hemisphere circulation were demonstrated in a number of studies \citep{Gnanadesikan2000,Wolfe2011}. We wish to identify the SO processes involved in this connection. First, we give a basic overview of Southern Ocean dynamics and theory to give context for our investigations before outlining modelling techniques used to investigate them. 


\section{Introduction to Southern Ocean dynamics}

As the Southern Ocean plays such a vital role in heat and carbon transport and the global overturning circulation there has been much interest over the years to understand it's complex dynamics and how the circulation might change in response to a changing climate. The SO is weakly stratified with very strongly tilted isopycnals across the ACC due to a lack of continental boundaries. This large isopycnal tilt supports a large geostrophic flow in the ACC and provides a route for deep water to move along isopycnals towards the surface. Many isopycnals outcrop and this allows for many density classes to feel direct buoyancy and momentum forcing from the atmosphere, allowing forcing of density classes well below the thermocline. 

Strong winds generate sloping isopycnals which are maintained by a lack of continental boundaries. Wind stress is around a maximum at 50$^o$S, which corresponds to a change in sign of wind stress curl ($\nabla \times \tau$). The Southern Hemisphere westerly $\tau$ is associated with equatorward Ekman transport, this leads to a surface divergence where the $\nabla \times \tau < 0$ and convergence where $\nabla \times \tau > 0$. Thus a weak vertical velocity shoals isopycnals to the South and depresses them to the North critical for the ACC and the overturning circulation. In the Northern Hemisphere mid-latitudes this is balanced by the development of a mean meridional velocity supported by a zonal pressure gradient, that relaxes isopycnal tilt. In the SO, there is no zonal pressure gradient to force a mean meridional flow. However some local anomalies in the zonal pressure gradient do arise and are crucial for  the ACC.

\subsection{Eddies in the the Southern Ocean}

Eddies play a significant and vital role in the Southern Ocean. In this thesis turbulent flows on spatial scales greater than or equal the \gls{Rsbrad} ($O$ 10km in the SO ) are referred to as mesoscale eddies, distinct from the mean flow by spatial and temporal perturbations rather than coherent eddies alone. Often referred to as a deviation from the time mean, denoted by a prime ($'$) symbol.

Oceanic eddies derive their energy primarily from baroclinic and barotropic instability which has an important impact on the large scale circulation. In principle, energy exchange between waves and mean flow can take place in both directions (to and from the mean flow).
%Figure here 
The energy exchange related to the Reynolds stress ($u'v'$) takes both signs, whereas the term related to the horizontal eddy density flux ($v'\rho '$) is predominantly positive so that energy is transferred from the mean flow to the waves, meaning that the mean flow is unstable to small perturbations. Two physically different mechanisms can be distinguished, corresponding to energy transfer, namely barotropic and baroclinic instability, both of which are important.
Sloping isopycnals act as a source of potential energy which can be released as kinetic energy. If two fluid parcels are exchanged and a denser fluid parcel sinks to be replaced with less dense fluid then kinetic energy is released through a vertical exchange or fluid parcels can move along a sloping path between isopycnals maximising at a slope of half that of the isopycnal slope in a slantwise exchange. The slantwise exchange normally transfers heat poleward. This occurs through baroclinic instability. 

Eddies can also be generated through horizontal shear in the horizontal velocity though barotropic instability. Barotropic instability is associated with the transport term: $$ -\frac{\partial U}{\partial y} <u'v'> $$ and can only occur if there is a horizontal shear of the background velocity i.e. \begin{equation*}
 -\frac{\partial U}{\partial y}  \neq 0 . 
\end{equation*} Since no vertical shear of U is needed (in contrast to baroclinic instability) a barotropic background current can produce the instability. Analysis of the energetics of the mean current shows that the energy is exchanged with the kinetic energy of the mean flow.

One of the important consequences of barotropic instability is that flows with small length scales are likely to be unstable. In fact, it can be shown that interacting short Rossby waves are unstable \citep{Gill1974a}, which helps to explain the turbulent nature of mesoscale motions in the ocean.

With both sloping isopycnals and strong barotropic flows, there is a very large amount of eddy activity in the SO playing an important role in both the ACC and the \gls{SO} overturning.

\subsection{Meridional Overturning Circulation}
\begin{figure}[H]
\centering
\includegraphics[width=0.8\textwidth]{../../Figures/SpeerSOMC.png}
\caption{Southern Ocean circulation schematic from \citet{speer2000}. Showing the main water masses and overturning circulations with the buoyancy forcing. Acronyms are defined in the glossary section for reference.}
\label{fig:SpeerSOMOC}
\end{figure}

One of the motivations of this work is understanding the overturning response to altered ocean physics to begin to understand how this might change in future climates. A general schematic is shown in \fref{fig:SpeerSOMOC} consisting of an upper and lower cell that is counter rotating \citep{Meredith2012}.  In the lower cell \gls{NADW} upwells near Antarctica and is converted to \gls{AABW} that sinks to the deep ocean. In the upper cell, the upwelling water is converted into \gls{AAIW} and \gls{SAMW} through the addition of heat and fresh water in the surface waters, which have a northward Ekman flow. The Upper cell is controlled by eddy and air-sea forcing \citep{rintoul2012southern} and this upper cell the main focus of the overturning in this thesis.

Strong polar westerlies drive a mean Ekman overturning circulation ($\overline{\Psi}$) that steepens isopycnals while baroclinic instability drives an opposing eddy-induced circulation ($\Psi ^*$) \citep{Marshall2003} and shown in eddy permitting and resolving models \cite{hallberg2006}. This partly compensates the ekman circulation leading to a small residual circulation that acts to advect tracers along mean isopycnals \citep{speer2000}:

\begin{equation}
\psi_{res}=\underbrace{\overline{\psi}}_{mean}+\underbrace{\psi ^*}_{eddy}.
\label{eq:res1}
\end{equation}
 
Exactly how eddies interact to contribute to the mean flow is outlined in \fref{sec:TEM}. Due to this residual nature of the Southern Ocean overturning the Meridional Overturning Circulation is referred to as the \gls{SOROC}. \citet{Toggweiler2008}, \citet{Marshall2003} and \citet{speer2000} showed that in the limit of weak interior diapycnal mixing the SO ROC transport must be along mean isopycnal surfaces and can only cross isopycnals in the surface diabatic layer and by deep convective mixing outside the Southern Ocean to connect the surface branch with the deep branch \fref{fig:SpeerSOMOC}. This limit does not apply to the deep cell of the SO ROC, associated with Antarctic Bottom Water formation, but is relevant for the pole-to-pole overturning circulation associated with North Atlantic Deep Water formation \citep{Wolfe2011}. This simple residual-mean decomposition allows for the parameterisation of eddies combining Gent-McWilliams and Green and Stone schemes \citep{Visbeck1997}. These parametrisations which have been refined over the years \citep[e.g.]{jansen2015}, but all assume largely adiabatic eddy fluxes. 

Some models show up to a 70\% disparity in models using full eddy compensation or no eddy compensation, which would have major implications for future anthropogenic CO$_2$ uptake \citep{lequere2007}. Making it of great interest

\subsection{ACC}

In contrast to gyre circulations strong zonal flows are parallel to wind forcing $\textbf{u} \cdot \, \tau _s$ generating the ACC, the Worlds longest and strongest current system, extending 24,000 km long and around 130$\pm $ 10 Sv \citep{whitworth1985}. Like much of Southern Ocean dynamics the ACC is also unique, barotropically flowing Eastward at almost all depths \footnote{Surface pressure gradient and stratification yields a positive shear to reduce the velocity with depth but, rarely enough to lead to a reversal.}%Cite?
. Although the  ACC is actually composed of filaments of jet streams bound by fronts, the ACC is often simplified to one flow that can be approximated in a model by channel domain with periodic flow in the zonal direction. From a momentum balance, \fref{eq:mombalance}, it can be seen that wind stress is balanced by bottom form stress that arises from correlation between zonal gradients in bottom topography and pressure anomalies.
\begin{equation}
\overline{\tau}^{x} _{wind} = p_b \frac{\overline{\partial \eta_b}}{\partial x},
\label{eq:mombalance}
\end{equation}  
where $\tau ^x _{wind} $ is the zonal wind stress, P$_b$ is bottom pressure and $\eta _b$ is the bottom topography. High pressure anomalies in SSH and are found upstream of bathymetric features and low pressure anomalies are found in lee of topography. The bottom form stress transfers momentum directly from the ocean into the solid earth. If topography is removed the ACC is seen to be roughly an order of magnitude larger than the observed values as first pointed out in \citet{munk1951}. 

\subsection{Eddy saturation.}

In recent years there has been much debate on the response of the ACC and SO ROC to increasing wind stress due a strong positive trend in the \gls{SAM} index \citep{thompson2002}, this was one of the strongest observed climatic trends \cite{Sallee2010}. However with longer time series of the SAM index and the inclusion of satellite observations show large inter annual and decadal variability suggest this trend is not as dramatic as it first appeared in the late 1990s \citep{Hogg2015}. None the less variations in response between the the ACC and the SO overturning to intensification of the polar westerlies allowed for a long of research into the mechanisms of eddies in the Southern Ocean and their role in Southern Ocean dynamics. With increasing wind stress there has been little response observed in isopycnal slope \citep{boning2008}. It is thought there is a near linear response in \gls{EKE} to increased \gls{symb:Tau}. Rather than accelerating the ACC transport, the momentum imparted by the wind is transferred to the bottom via interfacial form stress \cite{meredith2006}. \citet{morrison2013} showed a marked difference between eddy saturation and eddy compensation (ekman cancelling), indicating the differing depth scales for eddy driving of ACC and overturning can lead to differing responses to changing momentum forcing. Making the ACC response to altered forcing a worth while investigation as it may not have a similar response to the the SO ROC. 

\section{Residual Mean and Transformed Eulerian Mean Theory}

Due the the unique dynamics of the Southern Ocean the large scale mean circulation is influenced by the small scale time-varying components. Therefore time varying eddies need to be taken into account to establish the residual circulation rather than the fictitious deacon cell \citep{Doos1994}. Thus tend to approach SO dynamics in terms of residuals. This approach will be used through out this thesis and the background theory is outlined here.

\subsection{Assumptions and approximations:}
%% Explain why we need to make these !! (better justification)
\textit{Insert better justification here ....}
First we outline the major assumptions and approximations used in order to simplify the equations used.

\subsubsection*{Boussinesq approximation}

We assume a Boussinesq fluid (Volume is conserved) as variations in density with depth are just 2-3 \% , $ \rho _0 (z)$ becomes $ \rho _0$ allowing the Boussinesq equations:

\begin{equation}
\rho _0 \frac{Du}{Dt} = -2 \rho _0 \Omega \times u - \nabla \tilde{p} - \tilde{p} \nabla \Phi + \mathcal{F},
\label{eq:BQ1}
\end{equation}
\begin{equation}
\nabla \cdot u = 0,
\label{eq:BQ2}
\end{equation}
\begin{equation}
\rho _0 \frac{D (\theta, S)}{Dt} = ( \mathcal{G_S}, \mathcal{G_{\theta}}),
\label{eq:BQ3}
\end{equation}
\begin{equation}
\tilde{\rho} = F ( S, \theta, \rho _0 ) + F(S_0, \theta _0, \rho _0 (z)),
\label{eq:BQ4}
\end{equation}
where \ref{eq:BQ2} is the \gls{incompressibility} equation in an adiabatic system.  The symbols are defined in the list of symbols section in the glossary.
\subsubsection*{Small Rossby number}
Rossby number is small meaning the Coriolis frequency is important.
\begin{equation}
R_o=\frac{U}{f_o L} << 1 ,
\end{equation}
where $R_o$ is the Rossby number.
\subsubsection*{F-plane:}
The $\beta $ effect is small so that: 
\begin{equation}
\frac{\beta L}{f} \leq R_o
\end{equation}


\subsubsection*{Static stability}

\gls{Brunt Vaisala frequency} ($N^2$) is a function of depth (z) only so
\begin{equation}
N^2 (z) = \frac{\partial b}{\partial z} 
\end{equation}
where b is buoyancy.


\subsubsection*{Hydrostatic balance}
\begin{equation}
\frac{\partial \psi}{\partial z}=\frac{b}{f_o}
\end{equation}

\subsection*{Quasi-Geostrophy}
\label{sec:QG}
Assume \gls{QG} so there is no mean advection and no vertical component of eddy \gls{PV} flux. QG assumes small Rossby number, which we already assume along side:

\begin{itemize}
\item \textbf{Small aspect ratio:} Isopycnals have small slope as horizontal dimensions of ocean basins are much larger than vertical dimensions. Gives small vertical motions: \begin{equation}
\frac{\partial_x b}{\partial_z b}\, \& \,\frac{\partial_y b}{\partial_z b} \leq R_o 
\end{equation}
\item \textbf{Small Ek number:} Ek $<<$  1. 
\item stratified background state 
\item all effects of compressibility neglected
\end{itemize}

These assumptions allow us to write a geostrophic streamfunction:

The horizontal velocity divergence is 0.


\begin{equation}
\frac{\partial u}{\partial x} + \frac{\partial v}{\partial y} =0 .
\end{equation}
We can write:

\begin{equation}
u=-\frac{\partial \psi}{\partial y}, \quad v=\frac{\partial \psi}{\partial x}, \quad w=o ,
\end{equation}

where $\psi$ is the geostrophic stream function:
\begin{equation}
\psi = \frac{\rho -\rho_o(z)}{\rho_of_o}
\end{equation}

Now starting from the \gls{QG} equations:
\begin{equation}
\underbrace{D_g}_\text{time derivative}u-\beta yv - \underbrace{f_o v_a}_\text{ageostrophic velocities} = \underbrace{\mathcal{G}_x}_\text{External forcing on momentum},
\label{eq:qg1}
\end{equation}
\begin{equation}
D_gv-\beta yu - f_o u_a = \mathcal{G}_y,
\label{eq:qg2}
\end{equation}
\begin{equation}
\frac{\partial u_a}{\partial x} + \frac{\partial v_a}{\partial y} + \frac{\partial w_a}{\partial z} = 0,
\label{eq:qg3}
\end{equation}
\begin{equation}
D_g +N^2 w_a = \underbrace{\mathcal{B}}_{\mathclap{\text{nonconservativie buoyancy forces}}} - \frac{\partial \overline{v'b'}}{\partial y}.
\label{eq:qg4}
\end{equation}

Where $D_g$ is the material derivative: (

\begin{equation}
D_g=\frac{\partial}{\partial t} + u\frac{\partial}{\partial x} + v \frac{\partial}{\partial y}.
\end{equation}

Other symbols are define in the list of symbols).
The ageostrophic velocity is the difference between the actual velocity and the geostrophic one. The external forcing ($\mathcal{G}$) arises from wind, stress and friction etc. The non-conservative buoyancy forcing ($\mathcal{B}$) is from small scale mixing, and surface heat fluxes etc. 

Combining this using \gls{PV}:
\begin{equation}
q = f_o + \beta y + \frac{\partial v}{\partial x} - \frac{\partial u}{\partial y} + f_o \frac{\partial \frac{b}{N^2}}{\partial z}
\label{eq:PV}
\end{equation}
and substituting the \gls{QG} equations into: 
\begin{equation}
\chi = \frac{\partial \mathcal{G}_y}{\partial x}-\frac{\partial \mathcal{G}_x}{\partial y} + \frac{\partial u}{\partial y} + f_o \frac{\partial \mathcal{B}}{N^2}{\partial z}
\end{equation}

This gives the equation for Quasi-geostrophic Potential Vorticity (QGPV).
If  $\mathcal{G}=0$ and $\mathcal{B}=0$ then q is conserved (conservative flow). If flow is not conserved then $\chi$ represents local sources and sinks of q (viscous and diabatic effects).
\begin{equation}
D_g q= \chi
\label{eq:QGPV}
\end{equation}



\subsection{Eliassen-Palm theorem}

For small amplitude motions we can use:
\begin{equation}
\overline{u}(y,t)=\partial _y \overline{\psi}
\end{equation}
\begin{equation}
\partial _y \overline{b} = -f_o \partial _z \overline{u},
\end{equation}

then substituting into \fref{eq:PV}. We get the mean PV in terms of the geostrophic streamfunction:
\begin{equation}
\overline{q}=f_o + \beta y + \frac{\partial^2 \psi}{\partial y^2}+\frac{\partial^2 \psi}{\partial z^2}\frac{\partial \frac{f^2_o}{N^2}}{\partial z}.
\end{equation}

As $\psi$ can be decomposed into eddy and mean contributions:

\begin{equation*}
\psi '=\psi - \overline{\psi}
\end{equation*}

The eddy vorticity ($q'$) can be described as:

\begin{equation}
q'=f_o + \beta y + \frac{\partial^2 \psi '}{\partial y^2}+f \frac{\partial^2 \psi}{\partial z^2}\frac{\partial \frac{f^2_o}{N^2}}{\partial z}
\end{equation}

and the eddy velocity as:

\begin{equation}
v'=\frac{\partial \psi '}{\partial x} .
\end{equation}

So the eddy PV flux can be described by:

\begin{equation}
\overline{v'q'}= \frac{\partial (-\overline{u'v'})}{\partial y} + \frac{\partial \frac{f_o}{N^2}(\overline{v'b'})}{\partial z} .
\end{equation}

This can be simplified defining the Eliassen-Palms flux $\textbf{F}$ as:
\begin{equation}
\textbf{F}= \begin{pmatrix}
            -\overline{u'v'} \\ \frac{f_o}{N^2}\overline{v'b'}
            \end{pmatrix},
\end{equation}

where the meridional component is the negative of the zonal eddy momentum and the vertical is proportional to the meridional eddy buoyancy flux.

So the eddy PV flux can be written simply as:
\begin{equation}
\overline{v'q'}=\nabla . \textbf{F} .
\end{equation}

Now linearising the QGPV ~\fref{eq:QGPV} we get:

\begin{equation}
\frac{\partial q'}{\partial t} + \overline{u}\frac{\partial q'}{\partial x} + \overline{v}\frac{\partial \overline{q}}{\partial y}=\chi .
\end{equation}

Then we rearrange by multiplying by q' and averaging we get:

\begin{equation}
\underbrace{\frac{\partial q'^{2}}{2 \partial t} +  \underbrace{\cancel{\overline{\overline{u}q'}\frac{\partial q'}{\partial x}}}_\text{time av removes} + \overline{v'q'}\frac{\partial \overline{q}}{\partial y}}_\text{=0 when waves are steady}= \underbrace{\overline{\psi ' q'}}_{\mathclap{\text{ =0 if waves are conservative}}},
\end{equation}

which is the eddy enstropy equation. If waves are non divergent then $\nabla . F = 0$, meaning there is no eddy PV flux ($\overline{v'q'} = 0$).

This allows us to show eddies impacting the mean zonal circulation in the QGPV budget\footnote{For small, steady and conservative waves the eddy flux doesn't impact the mean flow (non acceleration theorem).}.:

\begin{equation}
\frac{\partial \overline{q}}{\partial t} + \frac{\partial \overline{v'q'}}{\partial y} = \overline{\psi} .
\end{equation}



To simplify the problem we introduce \gls{TEM} theory. 


\subsection{Transformed Eulerian Mean theory}
\label{sec:TEM}
Most of this theory is outlined in \citet{Marshall2003}. Starting from the QG equations outlined in ~\fref{sec:QG}. Eddy buoyancy and momentum fluxes change the mean zonal state through the terms $\frac{\partial (\overline{v'b'})}{\partial y}$ and $\frac{\partial (\overline{u'v'})}{\partial y}$ in equations \fref{eq:qg4} and \fref{eq:qg1}.\footnote{Sometimes there is no eddy momentum fluxes but thermal wind balance insists both momentum and buoyancy fluxes are required to change the mean flow so eddies will drive the ageostrophic terms as well inducing an ageostrophic mean motion.} 

If we redefine a mean meridional ageostrophic circulation.

Using ~\fref{eq:cont} and \fref{eq:qg3} defining the ageostrophic streamfunction as:

\begin{equation}
(\overline{v_a},\overline{w_a})=(\frac{\partial \overline{\psi} }{\partial y},\frac{\partial \overline{\psi}}{\partial z}).
\end{equation}

And putting back into the mean buoyancy budget \fref{eq:qg4}

\begin{equation*}
\frac{\partial\overline{b}}{\partial t} + \frac{\partial \overline{\psi}}{\partial z}N^2 + \frac{\partial (\overline{v'b'})}{\partial y} = \overline{\mathcal{B}},
\end{equation*}


assuming static stratification so that $N^2=N^2(z)$ and 
that $\frac{\partial \overline{v_a}}{\partial y} = - \frac{\partial \overline{w_a}}{\partial z} =\frac{\partial \overline{\psi}}{\partial y} $
Gives:

\begin{equation}
\frac{\partial\overline{b}}{\partial t} + \frac{\partial (\overline{\psi } N^2 +\overline{v'b'})}{\partial y} = \overline{\mathcal{B}}
\end{equation}

Moving $N^2$ out of brackets as it doesn't vary with latitude (static stability assumption):

\begin{equation}
\frac{\partial\overline{b}}{\partial t} + \frac{\partial (\overline{\psi} +\frac{\overline{v'b'}}{N^2})}{\partial y}N^2 = \overline{\mathcal{B}}.
\end{equation}

Here the term $\frac{\overline{v'b'}}{N^2}$ is the eddy flux term (giving a mean advection) so we can define $\psi ^* $ as the eddy induced mean stream function:

\begin{equation}
\psi ^* = \frac{\overline{v'b'}}{N^2}
\label{eq:eddyflux}
\end{equation}

And the residual stream function is defined as:

\begin{equation}
\underbrace{\psi _{res}}_\text{residual}= \underbrace{\overline{\psi}}_\text{ageostrophic} + \underbrace{\psi ^*}_\text{eddy}
\end{equation}

Now we can define a residual circulation:
\begin{equation}
(\overline{v_{res}},\overline{w_{res}})=(\frac{\partial \psi _{res}}{\partial y},\frac{\partial \psi _{res}}{\partial z})
\end{equation}

So if we now put $\psi _{res}$ back into the mean buoyancy budget \fref{eq:qg4}, the eddy terms fall out:

\begin{equation*}
\frac{\partial\overline{b}}{\partial t} + \overline{w_a}N^2= \overline{\mathcal{B}}- \underbrace{\frac{\partial \psi ^*}{\partial y}}_\text{$\psi _{res}$ - $\overline{\psi}$}N^2 
\end{equation*}

\begin{equation*}
\frac{\partial\overline{b}}{\partial t} + \overline{w_a}N^2= \overline{\mathcal{B}}- \underbrace{\frac{\partial \psi _{res}}{\partial y}}_\text{$\overline{w^*}$}N^2 -  \underbrace{\frac{\partial \overline{\psi}}{\partial y}}_\text{$\overline{w_a}$}N^2
\end{equation*}

\begin{equation*}
\frac{\partial\overline{b}}{\partial t} + \cancel{\overline{w_a}N^2}= \overline{\mathcal{B}}-\overline{w_{res}}N^2 - \cancel{\overline{w_a}N^2}
\end{equation*}

Which removes the eddy terms in the mean buoyancy budget:
\begin{equation}
\frac{\partial\overline{b}}{\partial t} + \overline{w_{res}}N^2= \overline{\mathcal{B}} 
\label{eq:mbbres}
\end{equation}

The equations of momentum ``transformed" :

\begin{equation}
\frac{\partial \overline{u}}{\partial t} - f_o \overline{v^*} = \overline{\mathcal{G}_x}+\underbrace{\overline{v'q'}}_\text{$\nabla . \textbf{F}$ eddy forcing}
\label{eq:momres}
\end{equation}
\begin{equation}
f_o \frac{\partial \overline{u}}{\partial z} = -\frac{\overline{b}}{\partial y}
\end{equation}
\begin{equation}
\frac{\partial \overline{v^*}}{\partial y} + \frac{\partial \overline{w^*}}{\partial z} = 0
\label{eq:cont}
\end{equation}
and the mean buoyancy budget
\begin{equation*}
\frac{\partial\overline{b}}{\partial t} + \overline{w^*}N^2= \overline{\mathcal{B}}.
\end{equation*}

So from the term $\nabla . \textbf{F}$ in \fref{eq:momres} showing that if the flow is conservative (non accelerational then there are no eddy influences!).

\subsubsection*{Momentum and buoyancy balances}

The mean buoyancy budget \footnote{Zonal average refer to stream wise averages in the presence of topography, so that eddy fluxes are transient.}
\begin{equation}
\frac{\overline{v}\partial \overline{b}}{\partial y} + \frac{\overline{w}\partial \overline{b}}{\partial z} + \underbrace{\frac{\partial \overline{v'b'}}{\partial y}}_\text{EP flux} + \frac{\partial \overline{w'b'}}{\partial y} = \underbrace{\frac{\partial B}{\partial z}}._\text{Buoyancy forcing due to air-sea interactions and small scale mixing}
\end{equation}
In terms of a residual circulation:
\begin{equation}
\psi _{res} = \overline{\psi} +\psi ^*
\end{equation}
\begin{equation}
\psi ^*=\frac{\overbrace{-w'b'}^\text{eddy flux}}{\underbrace{b_y}}_\text{mean meridional buoyancy gradient}
\end{equation}
Assuming that $\overline{v'b'}$ is in the $\overline{b}$ surface (assume all the eddy fluxes are along isopycnals) so that $\nabla . \overline{v'b'}$ can be written as fluxes along isopycnals as advective transport $\textbf{v}^* \nabla \overline{b}$:
\begin{equation}
u' \frac{\partial \overline{b}}{\partial x} + v' \frac{\partial \overline{b}}{\partial y} + w' \frac{\partial \overline{b}}{\partial z} 
\end{equation}
And replacing $\overline{v}$ and $\overline{w}$ with:
\begin{equation*}
\overline{w} = \frac{- \partial \psi_{res}}{\partial z} + \frac{- \partial \psi ^*}{\partial z} 
\end{equation*}
\begin{equation*}
\overline{v} = \frac{- \partial \psi_{res}}{\partial y} + \frac{- \partial \psi ^*}{\partial y} 
\end{equation*}
Then decomposing the eddy flux $(v'b',w'b')$ into along $\overline{b}$ components: $(\frac{\overline{w'b'}}{S_p},\overline{w'b'})$ and horizontal components $(\overline{v'b'}-\frac{\overline{w'b'}}{S_p},0)$  where:
\begin{equation}
S_p= \frac{-\overline{b_y}}{\overline{b_z}}.
\end{equation}
The buoyancy budget can be written in terms of a residual flux:
\begin{equation*}
(\frac{\partial \psi _{res}}{\partial y}) + (\frac{\partial \psi ^*}{\partial z}) \frac{\partial \overline{b}}{\partial y} + (\frac{\partial \psi ^*}{\partial z} -  \frac{\partial \psi _{res}}{\partial z})\frac{\partial \overline{b}}{\partial z} + \frac{\partial(\overline{v'b'}-\frac{\overline{w'b'}}{S_p},\frac{\overline{w'b'}}{S_p})}{\partial y}+ \frac{\partial (\overline{w'b'},0)}{\partial z}= \frac{\partial B}{\partial z}
\end{equation*}
If we define $\mu $ as:
\begin{equation*}
\mu=(\frac{\overline{w'b'}}{\overline{v'b'}})(\frac{1}{S_p})
\end{equation*}
We get:
\begin{equation*}
\underbrace{(\frac{\partial \psi _{res}}{\partial y}) \frac{\partial \overline{b}}{\partial y} + (\frac{\partial \psi _{res}}{\partial z})\frac{\partial \overline{b}}{\partial z}}_\text{Jacobian $J(\psi_{res} , \overline{b})$} = \frac{\partial B}{\partial z} - \frac{\partial (\overline{v'b'}-\frac{\overline{w'b'}}{S_p})}{\partial y}
\end{equation*}
And substituting in $\mu$ 
\begin{equation}
J(\psi_{res} , \overline{b}) = \frac{\partial B}{\partial z} - \frac{\partial (1-\mu )\overline{v'b'}}{\partial y}
\label{eq:mr03bbudget}
\end{equation}
When $\mu = 1$ then $\frac{\partial (0)}{\partial y}$ then the eddy flux is solely along $\overline{b}$ surfaces and there are no diapycnal fluxes (pure advection). (The interior = adiabatic so $\mu = 1 $). 

When $ \mu = 0 $ then diapycnal fluxes are important (e.g. in the mixed layer) so eddy fluxes are across buoyancy surfaces $ \overline{w'b'} = 0$. Mostly these surface fluxes are taken to be negligible and the adiabatic components are assumed to be the significant components. In the limit of adiabatic eddies, vanishingly small mixing and air-sea buoyancy fluxes ($\mu = B = 0$) so that:

\begin{equation}
\psi ^* = \frac{\overline{w'b'}}{\overline{b}_y}
\end{equation} 

to give:

\begin{equation}
\frac{\partial \psi _{res}}{\partial y}\frac{\partial \overline{b}}{\partial z} - \frac{\partial \psi _{res}}{\partial z}\frac{\partial \overline{b}}{\partial y}=0
\end{equation}

So $\overline{b}$ is advected by $\psi _{res}$.

Evaluating \fref{eq:mr03bbudget} at the base of the mixed layer \cite{Marshall2003} devised a well used diagnostic. Assuming that $\frac{ \partial b}{\partial z}= 0$ (vertical isopycnals) we can remove a term from the LHS:

\begin{equation*}
\cancel{\frac{\partial \psi _{res}}{\partial z}\frac{ \partial b}{\partial z}}-\frac{\partial \psi _{res}}{\partial z}\frac{ \partial b_0}{\partial y} = \frac{\partial B}{\partial z} - (1-\mu ) \frac{\partial}{\partial y}(\overline{v'b'}),
\end{equation*}

and integrate over the mixed layer to give:

\begin{equation*}
\int_{-h_{ml}}^{0} -\frac{\partial \psi _{res}}{\partial z} \frac{\partial b_0}{\partial y} \mathrm{d}z = \int_{h_{ml}}^{0} \frac{\partial B}{\partial z} - (1-\mu ) \frac{\partial}{\partial y}(\overline{v'b'}) \mathrm{d}z.
\end{equation*}

Evaluating at the base of the mixed layer; $\psi _{res}$ goes to zero at the surface. This allows us to write the LHS as the $ \displaystyle{\left. \psi_{res} \right |_{z=0} -\left. \psi_{res} \right |_{z=h_{ml}}}$

\begin{equation*}
0 - \left. \psi_{res} \right |_{z=h_{ml}} \frac{\partial b_0}{\partial y} = B_0 - (1- \mu) \int_{h_{ml}}^{0} \frac{\partial}{\partial y}(\overline{v'b'}) \mathrm{d}z
\end{equation*}

which is often condensed to:

\begin{equation}
\left. \psi_{res}\right|_{z=h_{ml}} \frac{\partial b_0}{\partial y} = \underbrace{\tilde{B_0}}_{\mathclap{\text{net buoyancy supplied to the mixed layer by air-sea fluxes and lateral adiabatic eddy fluxes}}}
\label{eq:MR03b}
\end{equation}

This allows for a simple diagnostic tool: If $\tilde{B} > 0$ (Buoyancy gain (heating)) then $\frac{\partial b_0}{\partial y} > 0$ so $ \left. \psi_{res}\right|_{z=h_{ml}} >0$. Relating surface buoyancy gain/loss to the sign of the SO ROC. Often we see this buoyancy gain/loss to be taken as simply the surface heat fluxes where the diabatic eddy contribution is assumed to be negligible. 

\section{TEM in The Surface Layers}

This original framework presented in \citet{Marshall2003} does not allow for real irreversible diapycnal eddy fluxes as Quasi-Geostrophy approximations are used. Which requires assumptions such as a small aspect ratio which ceases to be true in the surface mixed layer. When trying to establish the components of eddy fluxes we must consider how the surface mixed layer is handled in order to truly remove the along isopycnal component. In section \ref{sec:TEM} we use \fref{eq:eddyflux} which is taken from a Q-G case in \citet{Andrews1976} which is inconsistent with \fref{fig:MR03scheme} where we expect very steep isopycnals in the diabatic layer.  \fref{eq:eddyflux} would only be be sufficient in the interior where $\displaystyle{|\overline{b}_z|\gg |\overline{b}_y|}$ (strongly stratified). If there is any surface buoyancy gradient there will be diapycnal eddy fluxes and therefore vertical residual flux through the surface. To circumnavigate this issue \citet{Held1999,treguier1997} introduce a vanishingly thin surface layer in which the residual flow transitions to zero at the surface as used in the mixed layer diagnostic \fref{eq:MR03b} where the residual flux is assumed zero at the surface and only the base of the mixed layer is considered.   

In the mixed layer $\displaystyle{|\overline{b}_y|\gg |\overline{b}_z|}$ the eddy streamfunction becomes:
\begin{equation}
\Psi = \frac{\overline{w'b'}}{\overline{b}_y},
\label{eq:HS99}
\end{equation}
from \citet{Held1999}.
This gives horizontal fluxes in the mixed layers so no residual flux through the surface. However in this definition is not applicable in the interior where both the denominator and numerator in \fref{eq:HS99} may be very small leading to large errors. Thus it would be suggested to use  \fref{eq:eddyflux} in the interior and \fref{eq:HS99} in the mixed layer with the surface mass flux spread over the transition region. 

\citet{Plumb2005} suggest an alternative definition of $\Psi$:
\begin{equation}
\Psi = - |\nabla \overline{b}|^{-1} (\textbf{s}\cdot\overline{\textbf{u}'b'}),
\label{eq:PF05psi}
\end{equation}
and a residual diapycnal eddy buoyancy flux:
\begin{equation}
\textbf{F}(b) = \textbf{n}(\textbf{n}\cdot\overline{\textbf{u}'b'}),
\end{equation}
directed along mean buoyancy gradient. Where vector \textbf{n} is the mean buoyancy gradient ($\displaystyle{\sfrac{\nabla \overline{b}}{|\overline{b}|}}$) and vector \textbf{s} is along mean buoyancy surfaces ($\displaystyle{\textbf{n}\times \textbf{i} }$), which was implemented in \citet{kuo2005}. Where the buoyancy budget is rewritten:
\begin{equation}
\frac{\partial \overline{b}}{\partial t} + \overline{\textbf{u}}_{res} \cdot \nabla \overline{b} = - \textbf{n}(\textbf{n}\cdot\overline{\textbf{u}'b'}) - \overline{B}_z .
\label{eq:kuobb}
\end{equation}
Once again to evaluate such a budget the diabatic eddy heat flux is take as 0 at the base of the mixed layer .
 \begin{equation}
\int _{z_i}^{z_0} \left( \nabla \cdot (\chi _{res} \textbf{j} \times \nabla \overline{b} \right) + \int _{z_i}^{z_0} \nabla \cdot \textbf{F}(b) \mathrm{d}z = - B_0
\label{eq:kuobb1}
\end{equation}
From below the surface mixed layer the residual buoyancy transport and diabatic residual eddy buoyancy flux ( $\chi _{res}$ and $\textbf{F}(b)$ respectively) are negligible in an adiabatic interior. The surface buoyancy budged in a balance between the surface fluxes and horizontal buoyancy transports:
\begin{equation}
\frac{1}{y}\frac{\partial}{\partial y} \left[ y \int_{z_i}^{z_0} \left( \textbf{F}(b) + \chi_{res} \overline{b}_z \right) \right] = - B_0
\end{equation}
here \citet{kuo2005} assumes a non-zero vertical buoyancy gradient. Allowing residual eddy buoyancy fluxes to play a role in the surface heat budget alongside the residual buoyancy advection. 

\section{Motivation}

The resolution of current climate models is restricted by computational limitations, the average resolutions of coupled climate models is $\displaystyle{1 \deg}$ \cite{}, so they require parameterization of eddies. Thus the complex interaction of eddy processes in the Southern Ocean may not be fully captured in those models. These parameterization schemes do not represent diabatic eddy fluxes well, which could lead to missing feedbacks in global climate models. As a result, the debate is still open as to whether eddy parametrisations should be purely adiabatic or have additional diapycnal mixing terms \citep{Gent2011a}. It is of great interest to understand what possible feedbacks are being neglected in climate models and what the physical processes behind them are, thus we use idealised models to gain a better understanding of processes that are important when we consider the future response of the Southern Ocean. 

\citet{Marshall2003} showed that in most realistic cases the diabatic eddy contribution is often small compared to the surface forcing and that the SO ROC can be predicted from the surface forcing using \fref{eq:MR03b}. Many studies, however, show an interhemispheric connection via the Atlantic meridional overturning circulation (AMOC) and North Atlantic Deep Water (NADW) formation, which acts as a control on the SO ROC (e.g. ~\citet{Wolfe2009}, ~\citet{Nikurashin2012a}, and from the ~\cite{Marshall2003} schematic).  This suggests that the interhemispheric control on the SO ROC occurs via changes in the budget of Eq.~\ref{eq:mr03bbudget}. When investigating the SO ROC in a channel model the rest of the worlds ocean must be represented via a northern boundary condition, i.e. the stratification north of the channel, impacting the isopycnal slope in the interior of the channel. We wish to investigate the effects of that northern boundary condition on setting the SO ROC in a channel model and dynamics set by the northern boundary. 

\begin{figure}
\noindent \includegraphics[width=\textwidth]{../../Figures/TEM2.pdf}
\caption{Southern Ocean Overturning Circulation directed along isopycnals related to surface forcing and outside diabatic processes. Adapted from \protect{\citet{Marshall2003}} schematic.}
\label{fig:MR03scheme}
\end{figure}

Within the \citet{Marshall2003} framework we can see that the Southern Ocean is not disconnected from the rest of the worlds oceans, and must rely on outside diabatic processes to close the SO ROC see \fref{fig:MR03scheme}. We wish to investigate in a very idealised, theoretical, way the possible mechanisms and controls on the SO ROC. This will allow us to illuminate possible considerations that may be overlooked when setting up channel models and glimpse possible interactions with boundary conditions that may have been previously unconsidered.


% Model overview and Set up 
\input{setup.tex}
% Chapter 2
\chapter[Diabatic eddies]{Diabatic eddies setting buoyancy forcing}
\label{chap:2}


This chapter focuses on the SO ROC response to closing the northern boundary condition in a highly idealised setting to understand the most basic fundamentals of this process and establish if in fact the SO ROC vanishes with a closed northern boundary.

\section{Introduction}
%%

Changes to the Southern Ocean Residual Overturning Circulation (SO ROC) could have large effects on ocean circulation and climate, making it of great interest to understand what sets the strength and sense of the SO ROC. The upper cell of the SO ROC is closed outside the Southern Ocean connecting the surface branch with the deep branch (Fig.~\ref{fig:BBschemandmodel}).

\begin{figure}
\noindent \includegraphics[width=\textwidth]{../../Figures/Diagram_Sept16.pdf}
\caption{An illustration of the channel model configuration, showing the surface forcing and the sponge layer at the northern boundary, with the expected residual circulation from the surface forcing at short sponge relaxation time scales depicted. The SO ROC in coloured arrows is determined in the surface mixed layer, which in this configuration can be several hundred meters deep, and directed along mean isopycnals in the interior.}
\label{fig:BBschemandmodel}
\end{figure}

We look at this within a modelling context where a Northern boundary condition must be applied in order to close the SO ROC. In the framework of \cite{Marshall2003} the SO ROC can be related to the effective buoyancy forcing in the Southern Ocean channel. However, without an open boundary the SO ROC cannot be maintained, implying that the effective buoyancy forcing in the Southern Ocean must go to zero. This part of the SO ROC circulation must be represented when considering the SO in isolation, \citet{Marshall2003} argued that in the present climate the effective surface buoyancy forcing in the Southern Ocean is dominated by surface buoyancy exchange between ocean and atmosphere. If the Northern boundary of a channel model is closed then the buoyancy budget in the Southern Ocean needs to change. In a set-up of fixed surface fluxes and no diapycnal mixing (consistent with the adiabatic pole-to-pole circulation), either convective mixing, or diabatic eddy fluxes must increase to cancel the surface fluxes. This suggests the choice of northern boundary condition may have a large impact on the dynamics of the model impacting the outcome of any theory investigated. 

Here we start in a very idealised theoretical framework considering the closure of the northern boundary. Specifically we show that the response in diabatic forcing in the mixed layer is achieved by diabatic eddy fluxes altering the effective surface buoyancy forcing. This is used as the initial starting point in our investigation before proposing other factors to investigate. Our first line of investigation is to establish if indeed the SO ROC ceases to exist when we close the northern boundary without altering the surface forcing and if that change is a step change or a smooth transition. 
%%%%%%%%%%%%%%%%%%%%%%%%%%%%%%%%%%%%%%%%%%%%%%%%%%
%%                   Theory                     %%
%%%%%%%%%%%%%%%%%%%%%%%%%%%%%%%%%%%%%%%%%%%%%%%%%%
%%
\section{Theory}

We investigate the overturning adjustments associated with changes in diabatic processes outside the Southern Ocean by considering an idealised channel model in which diabatic processes further north are represented through a diabatic sponge layer at the northern side of the channel. The diabatic forcing of the sponge layer is decreased by increasing the specified relaxation time scale in which the stratification is restored to a prescribed profile. When the northern boundary condition is changed, we expect a response in the diabatic eddy fluxes, adjusting the net diabatic forcing in the surface mixed-layer to accommodate the change in diabatic forcing, and as a result in stratification, at the northern boundary. Also, because the unchanged winds will leave the Eulerian MOC unaffected, a change in the SO ROC must be accommodated by a change in the eddy-driven overturning. So, the change in eddy-driven overturning must be associated and consistent with a change in net buoyancy forcing in the diabatic mixed-layer.
Here, we focus on how the time-mean heat budget is affected. To simplify the analysis we consider a model setup with a linear equation of state, equating buoyancy to temperature. Starting from the time mean heat budget we write:

\begin{equation}
\nabla \cdot \overline{\textbf{u}T} = \frac{\partial Q}{\partial z},
\label{eq:B_budget}
\end{equation}

where Q is the heat supplied to the surface through air-sea fluxes and other diabatic processes like diapycnal mixing and convective overturning.

Starting from this time mean heat budget \citet{Marshall2003} created a widely used diagnostic by decomposing the eddy fluxes into adiabatic and diabatic components and removing the diabatic eddy fluxes from the LHS of Eq.~\ref{eq:B_budget} to the RHS. The background theory for this is described this in detail in section \ref{sec:TEM} where we arrive at \fref{eq:mr03bbudget}. Evaluated at the base of the mixed layer, below which the SO ROC is constant along mean isopycnals, the SO ROC is related to the diabatic forcing over the mixed layer. Putting \fref{eq:mr03bbudget} from \cite{Marshall2003} in terms of temperature alone, as per our model's equation of state gives:

\begin{equation}
 \Psi_{res} |_{hml} \frac{\partial T}{\partial y} =  Q - \int_{hml}^{0} \frac{\partial \left(\overline{v'T'} - \frac{\overline{w'T'}}{S_T} \right)}{\partial y } \mathrm{d}z,
 \label{eq:MR03}
\end{equation}

 where $\Psi_{res}$ is the SO ROC and $S_T=-\overline{T_y}/\overline{T_z}$ is the slope of mean temperature surfaces.~\citet{Marshall2003} showed that in most realistic cases the diabatic eddy contribution and other diabatic forcing terms are small compared with the surface forcing and that the SO ROC can be predicted from $Q=Q_{surflux}$, neglecting diapycnal mixing and convective overturning. As our motivation for this study is the idea of far field forcing altering the SO ROC we look to the role of northern boundary conditions in channel models.
 For a different stratification north of the ACC, the diabatic eddy component may become as important as the surface buoyancy forcing term, or deeper convective mixing in the SO channel must come into play. We will assume that the diapycnal mixing away from the bottom will remain negligible and will not affect the SO ROC upper cell and that diabatic eddy fluxes cannot be neglected. This could mean when setting up a channel model to evaluate the SO ROC response to altered surface forcing the fundamental dynamics may have been determined by the boundary condition applied to maintain a 3 cell SO ROC.

In some scenarios the diabatic eddy contribution to the heat budget may not be negligible and must be considered. To do this Eq.~\ref{eq:MR03} must be reformulated to become compatible with the model equations. Part of this reformulation is due to the model's discretization, but another part is more fundamental. Some parts of the TEM theory used in \citet{Marshall2003} use QG assumptions that are not appropriate in the surface mixed layer. When deriving \fref{eq:MR03} the QG stream function is represented by \fref{eq:eddyflux} which means the left hand side of \fref{eq:MR03} is appropriate for the base of the mixed layer and below. This makes the original TEM framework inappropriate to investigate the mixed layer.  If we keep in a Cartesian framework and start from Eq.~\ref{eq:B_budget} and making a Reynolds decomposition the budget is split into eddy and mean transport terms: 

\begin{equation}
\frac{\partial \overline{v}\overline{T}}{\partial y } + \frac{\partial \overline{w}\overline{T}}{\partial z } + \frac{\partial \overline{v'T'}}{\partial y } + \frac{\partial \overline{w'T'}}{\partial z } = \frac{\partial \overline{Q}}{\partial z}.
\end{equation}

Traditionally the eddy fluxes are decomposed  into an along $\overline{T}$ component and a remaining horizontal component ~\citep{Marshall2003}. Here we decompose the transport terms into an along $\overline{T}$ component and a remaining vertical component:
 
\begin{equation}
\left(\overline{v'T'},\overline{w'T'}\right) =  \underbrace{\left(\overline{v'T'}, \overline{v'T'}S_T\right)}_\text{along $\overline{T}$}  + \underbrace{\left(0,\overline{w'T'} - \overline{v'T'}S_T \right)}_\text{remaining vertical}.
\end{equation}

Diabatic eddy fluxes should be down the mean buoyancy gradient. Decomposing the eddy fluxes along isotherm and either a remaining vertical (done here) or horizontal \citep[as in]{treguier1997}. The two different decompositions makes differing assumptions about the mixed layer, the assumption in the tradition TEM configuration is as in \fref{fig:mixedlayer}.a whereas our assumption assumes a layer like \fref{fig:mixedlayer}.b  \citep{Plumb2005}.
\begin{figure}
\center
\noindent \includegraphics[width=0.7\textwidth]{../../Figures/mixedlayer.pdf}
\caption{Mean isotherms in a hypothetical mixed layer (grey) showing a) assumed no vertical temperature gradient with purely horizontal diabatic fluxes and b) assumes isotherms are not vertical and an artificial thin layer where there is no along isothermal flow. \protect{ adapted from \citet[]{Plumb2005}}  }
\label{fig:mixedlayer}
\end{figure}

We have chosen to combine eddy flux vectors this way to make a direct comparison possible between diabatic forcing by surface heat fluxes and by diabatic eddies in the surface mixed layer:

\begin{equation}
\underbrace{\frac{\partial \overline{v}\overline{T}}{\partial y } + \frac{\partial\overline{w} \overline{T}}{\partial z } + \frac{\partial \overline{v'T'}S_T}{\partial z } + \frac{\partial \overline{v'T'}}{\partial y }}_\text{Adiabatic  advective fluxes ($\nabla \cdot \overline{\textbf{u$_{\textbf{A}}$}T}$)} = \underbrace{\frac{\partial Q}{\partial z}}_\text{air-sea fluxes} - \underbrace{\frac{\partial \left( \overline{w'T'}-\overline{v'T'}S_T \right)}{\partial z }}_\text{Diapycnal fluxes} = \frac{\partial (Q - D)}{\partial z},
\label{eq:cart}
\end{equation}
with
\begin{equation}
D = \overline{w'T'}-\overline{v'T'}S_T 
\end{equation}

It is noted in \citet{Plumb2005} that these "Raw" eddy fluxes contain reversible diapycnal mixing as well as irreversible mixing and will typically be much larger than the true irreversible diabatic eddy fluxes. The L.H.S will only represent the residual flux divergence when the the flow is adiabatic below the surface mixed layer. In a Cartesian frame work these terms acts as approximations for the heat budget of diabatic eddies and adiabatic fluxes but will differ from the residual mean counter parts that require additional surface mixed layer assumptions (discussed in chapter 1) in order to calculate. When we refer to diabatic eddy heat flux divergence we acknowledge by this definition additional reversible fluxes will be included so these fluxes will be quantitatively larger than the true irreversible diabatic fluxes.

It should be stressed that the diabatic surface mixed layer, is the layer over which Q and D go to zero. This layer equals the depth of the (winter) convective mixing or (winter) mixed layer. In our model seasonality is excluded, but the mixed layer depth can still be O (100 m), or even O (1000 m) in some of the experiments we will discuss later in this chapter.

Tracer evolution by the SO ROC is purely advective (non divergent), so that:

\begin{equation}
\overline{\nabla \cdot \textbf{u}T} = \overline{\textbf{u}} \cdot  \nabla \overline{T} ,
\label{eq:adv}
\end{equation}
This arises from the continuity equation:
\begin{equation}
\overline{\nabla \cdot \textbf{u}} = 0 ,
\label{eq:cont}
\end{equation}
In continuous equations if \fref{eq:cont} is true then \fref{eq:adv} must be true. This however, does not hold in a discrete model due to tracer and velocity grid alignment. In the MITgcm the part advection scheme FORTRAN routine generalises 1-D advection to 3-D by removing the local divergence flow separately in each dimension. Using the identity:

\begin{equation}
v \frac{\partial Q}{\partial y} =\frac{\partial v Q}{\partial y} - Q \frac{\partial v}{\partial y},
\end{equation}
This indicates on a discrete model grid \fref{eq:adv} would have to be written:
\begin{equation}
\overline{\nabla \cdot \textbf{u}T} = \overline{\textbf{u}} \cdot  \nabla \overline{T} - \overline{T} \cdot \textbf{u} ,
\label{eq:advdiscrete}
\end{equation}
This has implications for transferring the continuous equations from \cite{Marshall2003} to discrete equations. As a result, the divergence of heat transport cannot be formulated as advection by a (residual) stream function adequately in discrete equations. For this reason we will use the heat budget of Eq.~\ref{eq:cart} to diagnose the contribution by the residual circulation, diabatic eddies, surface heat fluxes, and remaining terms. The LHS of \fref{eq:cart} is the advective transport and the RHS is the diapycnal divergent terms. The LHS will disappear when the circulation is adiabatic but however will be larger than the true irreversible diabatic transport.

From the derivation of the buoyancy budget put in a TEM frame work it can be noted that if the local buoyancy forcing is kept constant, in order for the heat budget (Eq.~\ref{eq:cart}) to hold, there must be a response in the diabatic eddy heat flux if the residual circulation changes in response to changing northern boundary conditions. This is illustrated in Fig.~\ref{fig:BBschemandmodel} where the surface temperature gradient is related to the northern boundary stratification through geometrical arguments:
\begin{equation}
T_(y) = T_N(z=yS_T),
\end{equation}
so that
\begin{equation}
\frac{\partial T}{\partial y} =  -S_T \left(\frac{\partial T_N}{\partial z}\right),
\label{EQ:nbc_geom}
\end{equation}

%%%%%%%%%%%%%%%%%%%%%%%%%%%%%%%%%%%%%%%%%%%%%%%%%%
%%                Model Set up                  %%
%%%%%%%%%%%%%%%%%%%%%%%%%%%%%%%%%%%%%%%%%%%%%%%%%%
%%
\section{Model Setup}
We use an idealised channel model setup similar to ~\citet{Abernathey2011} and ~\citet{Zhai2014}. The model is based off the set up described in \fref{sec:Setup_stand} and we give a brief overview of the exact set up used here. Key model parameters are outlined in Table~\ref{tab:setup1} and the setup is shown schematically in Fig.~\ref{fig:BBschemandmodel}.

\begin{table}
\caption{Model Setup for flat bottom fixed-surface flux experiments.}
\label{tab:setup1}
\begin{center}
\begin{tabular}{|c|c|c|}
\hline \hline
\textbf{Symbol} & \textbf{Description}  & \textbf{Value} \\ 
\hline 
L$_x$, L$_y$, H & Domain & 1000 km, 2000 km, 2985 m \\ 
\hline 
L$_{sponge}$ & Length scale of sponge layer & 100 km \\ 
\hline 
Q$_0$ & Surface heat flux magnitude maximum & 10 W m$^{-2}$ \\ 
\hline 
$\tau _0$ & Max surface wind stress & 0.2 N m$^{-2}$  \\ 
\hline 
dx, dy & Horizontal grid spacing & 5 km  \\ 
\hline 
dz & Vertical grid spacing & 10-280 m \\ 
\hline 
Adv Scheme & 7$^{\text{th}}$ order centred & 7 \\ 
\hline 
$\tau_{R}$ & Sponge relaxation time scale & 3-day - $\infty$ \\ 
\hline 
r$_b $& Linear bottom drag parameter & 1.3x10$^{-3}$ m s$^{-2}$ \\ 
\hline 
\end{tabular}
\end{center}
\end{table}

In modelling studies of the Southern Ocean, sponge layers can be used to provide a northern boundary condition representing outside processes that would maintain a background stratification, enabling the SO ROC to close \citep{Abernathey2011,Abernathey2014,Zhai2014}. By altering the relaxation rate of the sponge layer we can show how important the choice of the northern boundary condition is. We do this by changing the relaxation time scale $\tau_{T}$ of the sponge layer to impose a northern boundary condition of varying rigidity.

The northern boundary condition is altered by changing the relaxation time scale $\tau_{T}$ from days to years to decades to an infinite time scale (closed northern boundary). Short relaxation time scales strongly constrain the northern boundary stratification and the SO ROC to a circulation effectively determined by that stratification. Increased relaxation time scales provide only weak constraints allowing the stratification to free evolve leading to a breakdown of the SO ROC. When $\tau_T$ is set to infinity the northern boundary acts like a closed wall.

Each model run is spun up for 400-3000 years (depending on relaxation time scale) to reach equilibrium indicated by mean Kinetic Energy and SO ROC strength (significantly longer than \citet{Abernathey2011} due to the longer relaxation time scales used here). The model is then run for a further 100 years with results averaged over this period \footnote{required to close the heat budget}.

%%
%%%%%%%%%%%%%%%%%%%%%%%%%%%%%%%%%%%%%%%%%%%%%%%%%%
%%            Overturning Response              %%
%%%%%%%%%%%%%%%%%%%%%%%%%%%%%%%%%%%%%%%%%%%%%%%%%%
%%
\section{Overturning response}

The motivation for this study is a) the conjecture that the SO ROC vanishes when the northern boundary is closed, irrespective of the surface forcing and b) %% Hdthe asph forward. At B ??
that the SO ROC can be determined for all local forcing
%%Ho of e consider hold 
as we keep local fluxes constant but keep change the northern boundary condition. Before moving to a detailed evaluation of the processes involved, we test whether this is the case. A series of experiments were run varying the relaxation time scale $\tau_T$ in Eq.~\ref{eq:sponge} from 3 days to a fully closed channel ($\tau_T = \infty$). The relaxation time scales used were: 3, 10, 30, 100, 1000, 3000, 10000 days and no relaxation ($\tau_T = \infty$). In this section, we show the results from 4 runs: a sponge layer with relaxation timescale of 3 ,300 and 300 days and a fully closed boundary. To allow us to establish if this is a stepped response or a smooth transition to SO ROC collapse.

\begin{figure}[H]
\noindent \includegraphics[width=\textwidth]{../../Figures/MOC1.pdf}
\caption{The Eulerian mean streamfunction ($\overline{\Psi}$), calculated from 100 years averaged velocities. }
\label{fig:MOC}
\end{figure}

The Eulerian mean overturning stream function ($\overline{\psi}$) remains constant \fref{fig:MOC} (maximum $\approx$ 2.25 Sv, which scales to $\approx$ 60 Sv for a full channel) regardless of the northern boundary condition. This is expected as the Eulerian component is wind driven and throughout all our experiments the surface wind stress remains constant. We will not continue to show the Eulerian mean overturning in subsequent runs unless it differs from \fref{fig:MOC}.

The SO ROC is calculated as an isothermal stream function\footnote{A very good approximation for the residual stream function, but we note that this deviates from the complete ROC in the surface layers.} following the method of \citet{Abernathey2011}:
\begin{equation}
\psi_{res}(y, T)= \frac{1}{\Delta t} \int_{t_o}^{t_{o}+\Delta t} \int_{0}^{L_x}  \int_{T}^{0} vh \,\mathrm{d}T \,\mathrm{d}x\,\mathrm{d}t,
\label{eq:psidense} 
\end{equation} 
where $\displaystyle{h = \frac{-\partial z}{\partial T}}$ is the layer thickness in potential temperature (T) and the averaging period $\Delta t$ is 100 years. The isothermal stream function is calculated over 42 discrete potential temperature layers shown in \fref{fig:RemapT1} and is remapped onto depth coordinates to give $\psi_{res}(y,z)$ shown in Fig.~\ref{fig:MOC+Remap}.
When a short relaxation time scale of 3 days is applied the ROC features 3 distinct cells directed along mean isopycnals, reproducing the result of \cite{Abernathey2011}. The SO ROC has a maximum of $\pm$ 0.75 Sv (30$\%$ of the mean) away from the surface. This is a realistic result as scaled to a full-length channel (approximately a factor of 25 times larger), a transport of 15 Sv would be obtained. This three cell structure disintegrates with increasing relaxation time scale and becomes virtually zero below the surface mixed layer when the northern boundary is closed (Fig.~\ref{fig:MOC+Remap}.d. 

\begin{figure}[H]
\noindent \includegraphics[width=\textwidth]{../../Figures/ROCyz1.pdf}
\caption{The isothermal stream function $\Psi_{res}(y,\theta)$ remapped onto depth coordinates, to give $\Psi_{res}(y,z)$ for a) $\tau _R$ = 3 day and b) no relaxation. Isotherms in multiples of 1$^{\circ}$C are overlaid as solid black contours. The surface heat forcing is displayed above.}
\label{fig:MOC+Remap}
\end{figure}

\begin{figure}[H]
\noindent \includegraphics[width=\textwidth]{../../Figures/ROCyT1.pdf}
\caption{The isothermal stream function $\Psi_{res}(y,T)$ for a) $\tau _R$ = 3 day and b) no relaxation.}
\label{fig:RemapT1}
\end{figure}

When the northern boundary of the Southern Ocean is closed, there is a complete collapse of the SO ROC and very weak diabatic circulations remain that are completely confined to the surface mixed layer i.e. the depth of the convective overturning in the model. Below the surface mixed layer no circulation connected to the northern boundary can be maintained as there is no longer a means to close such a circulation away from the adiabatic interior. This underscores the SO ROC being part of an adiabatic pole-to-pole circulation \citep{Wolfe2011}. When the circulation at the northern end collapses, the circulation at the southern end is doomed to disappear as well. 

Fig. \ref{fig:MOC+Remap} shows that changing the northern boundary condition completely alters the domain-wide stratification and the diabatic layer-depth is increased with increasing $\tau _R$. Short relaxation time scales show the top-to-bottom stratification of today's Southern Ocean, while longer relaxation time scales set up an alternate state with a large stratified surface layer and a sharp internal boundary layer (Fig. \ref{fig:MOC+Remap}. c-d). The northern boundary stratification T$_N$ also plays an important role in setting the isopycnal slope in the interior of the domain, which can be associated with a meridional buoyancy gradient allowing for the SO ROC. This result implies that not only diabatic eddy fluxes may cancel the surface heat flux in the mixed-layer, but also the role of convective mixing becomes more prominent and might be instrumental in collapsing the net surface forcing over the mixed layer. These results suggest that the SO ROC must be connected to the north of the domain in order to be maintained.

\begin{figure}
\noindent \includegraphics[width=\textwidth]{../../Figures/Closed_boundary_profile_ROC_300day.pdf} 
\caption{The isothermal stream function $\Psi_{res}(y,\theta)$ remapped onto depth coordinates, to give $\Psi_{res}(y,z)$ for a sponge layer relaxing to the closed boundary profile with a relaxation time scales of 300 days. Isotherms in multiples of 1$^{\circ}$C are overlaid as solid black contours.}
\label{fig:closedprofile}
\end{figure}

A question that arises is what is the cause of this collapse? Is it the change in stratification or the decrease in diabatic forcing in the sponge-layer? As a test we ran an experiment where the stratification is restored on a three-day timescale towards the profile displayed in the closed boundary run (Fig.~\ref{fig:MOC+Remap}.d . With a medium 300-day relaxation time scale, the isotherm streamfunction produced is shown in Fig.~\ref{fig:closedprofile}. The stratification becomes inconsistent with the diabatic forcing in the sponge-layer, implying an unstable solution. A small perturbation from the prescribed profile due to an eddy, invokes a diabatic forcing term that drives the circulation further away from the prescribed profile. Using shorter relaxation time-scales (3 days) results in an equally strong oscillating ROC that does not reach a steady state. These oscillations have peak to peak variations in total kinetic energy of 40$\%$ of the mean value, one to two orders of magnitude larger than in other spin-up runs.
We conclude that the SO ROC is associated with net diabatic forcing and that the stratification associated with a collapsed SO ROC is inconsistent with net diabatic forcing. We address the impact of altering the boundary stratification and surface forcing systematically in chapter \ref{chap:3}.
% inclused oscillation EKE plot?
%mention the difficulty in setting up such a scenario and how they basically are inconsitent forcing that can not work
%%
%%%%%%%%%%%%%%%%%%%%%%%%%%%%%%%%%%%%%%%%%%%%%%%%%%
%%              Buoyancy Budget                 %%
%%%%%%%%%%%%%%%%%%%%%%%%%%%%%%%%%%%%%%%%%%%%%%%%%%
%%

\section{Heat budget}

\begin{figure}
\noindent \includegraphics[width=\textwidth]{../../Figures/neefig4.png} 
\caption{The components of the full depth buoyancy budget evaluated in as in Eq.~\ref{eq:cart}. The advective transport component is show in black and diapycnal transport in blue, surface heat forcing in red and the total in thin black line.}
\label{fig:tembb_fix}
\end{figure}

To assess if a collapse in the SO ROC is at least partly due to a response in the diabatic eddy heat flux divergence we now move to plotting the terms of the heat budget \fref{eq:cart}. The terms of Eq.~\ref{eq:cart} are plotted in Fig.~\ref{fig:tembb_fix}. The difference between the thin grey line and the red line indicates the role of convective mixing and diabatic processes in the sponge layer. We also have included free surface correction term required due to the linear free surface boundary condition used in the model. This is calculated by:
\begin{equation}
\mathcal{T}_{surfcorr} = \frac{WTHMASS}{dz},
\label{eq:surcor}
\end{equation}
where WTHMASS is the vertical mass-weight transport of potential temperature at the surface. Although included in the thin black line term to assess how well the heat budget closes, this term is $\mathcal{O}10^{-7}$, so essentially negligible being 1-2 orders of magnitude less than our flux divergence terms.
We plot the LHS of \fref{eq:cart} the advective transport head divergence term as the SO ROC heat divergence (below the surface diabatic layer) in black in \fref{fig:tembb_fix} and the down gradient eddy heat heat flux divergence as our diabatic eddy term in blue, acknowledging these terms are larger than the real ROC/ diabatic eddy divergence terms. In all cases diabatic eddy heat fluxes nearly cancel the heat transport by the adiabatic ROC, with the surface heat flux being a small residual. These results suggest that for this set up there is no regime where diabatic eddy heat fluxes can be neglected in the budget of Eq.~\ref{eq:MR03}. In the northern 75\% of the domain the heat transport by the ROC gradually decreases with increasing restoring time scale, until in the closed basin case heat transport divergence by the ROC drops to zero and the diabatic eddy fluxes almost exactly cancel the surface forcing. In the southern part of the domain a surface confined ROC cell remains and here the diabatic eddy fluxes at first order cancel the heat transport accomplished by this residual cell. Changes in convective mixing are negligible in affecting the net surface forcing over the surface mixed layer. Figure \ref{fig:tembb_fix} highlights the changing diabatic eddy heat flux divergence and indicates the role of diabatic eddies in redistributing heat. However, as this figure shows the depth integrated divergence, we do not see the full spatial pattern. To investigate the distribution of the diabatic eddy heat flux divergence with depth, we plot the zonally averaged diabatic eddy heat flux divergence in latitude-depth space revealing the full extent of the diabatic eddy heat flux divergence changes (Fig.~\ref{fig:dhd}). 

\begin{figure}
\noindent \includegraphics[width=\textwidth]{../../Figures/dhdnew.png} 
\caption{Zonal mean diapycnal heat divergence for relaxation timescale 3, 300, 3000 days and no relaxation.}
\label{fig:dhd}
\end{figure}

The figure reveals that the magnitude of the diabatic eddy heat flux divergence/convergence is increasing almost everywhere with increasing relaxation timescales. The diabatic eddy heat flux divergence/convergence disappears where there is adiabatic flow, however this is not the true diabatic eddy heat flux convergence simply the down temperature gradient diapycnal fluxes. Large changes in the heat redistribution occur in the upper 1250 m. The depth over which the diabatic eddy fluxes are significant increases with increasing mixed layer depth as the relaxation time scale increases. Also the magnitude of the heat flux divergence increases, but at the bottom of the mixed layer a dipole pattern arises, which contributes little to the integral over depth. The diabatic eddy heat fluxes remove heat from the bottom of the mixed layer and transport it to the middle of the mixed layer layer, allowing for the creation of a sharp internal boundary layer, as seen in Fig.~\ref{fig:MOC+Remap}b, and a deep surface mixed layer. Heat transport by the ROC (\fref{fig:rhd}) is doing the opposite, bringing the heat downwards. At the southern end, the diabatic eddies cool everywhere, opposing the heating by the surface confined ROC. At the northern end, a large area arises with weak heat convergence in the upper part of the mixed layer, with heating at the bottom due to diabatic eddies, opposing cooling by the ROC. Although the SO ROC nearly collapses, within and across the surface mixed layer a weak cell remains that is associated with non-negligible heat transport due to the increased vertical gradient in temperature. From these figures it is clear that changes in the northern boundary condition lead to large changes in the effective surface heat forcing through a diabatic eddy response. To further understand this adjustment we must consider what sets the strength of the diabatic eddies.   
\begin{figure}
\noindent \includegraphics[width=\textwidth]{../../Figures/rocdiv1.pdf} 
\caption{Zonal mean Adiabatic heat divergence for relaxation timescale 3, 300, 3000 days and no relaxation.}
\label{fig:rhd}
\end{figure}

It should be noted that the increase in diabatic eddies is accompanied by a decrease of the vertically integrated diabatic eddy heat flux convergence, compare Figs.~\ref{fig:tembb_fix}d and \ref{fig:dhd}d. This occurs as the diabatic eddy heat flux convergence appears to be more and more as a perfect dipole pattern when the diabatic forcing in the sponge layer decreases. After establishment of an internal boundary layer, the diabatic eddies are associated with heating and equally large cooling in the upper and lower half of the internal boundary layer, respectively, the surface-confined ROC is associated with the opposite pattern. 

%%
%%%%%%%%%%%%%%%%%%%%%%%%%%%%%%%%%%%%%%%%%%%%%%%%%%
%%         Controls on Diabatic Eddies          %%
%%%%%%%%%%%%%%%%%%%%%%%%%%%%%%%%%%%%%%%%%%%%%%%%%%
%%
\section{Controls on diabatic eddies}
%%%%%%%%%%%%%%%%%%%%%%%%%%%%%%%%%%%%%%%%%%%%%%%%%%
%%                                              %% 
%%  Eddy energetics must be discussed here.     %%
%%       Refer to Olbers and Eden book          %%
%%              Page 377 of pdf                 %%
%%%%%%%%%%%%%%%%%%%%%%%%%%%%%%%%%%%%%%%%%%%%%%%%%%
Having found a significant response in the diabatic eddy heat flux divergence, it is now important to understand what establishes this response. 
The controls on diabatic eddy flux divergence can be thought of as being made up of an eddy velocity perturbation $v'$ , and a temperature perturbation $T'$. A relative measure of $v'$ can be obtained by taking the square root of the eddy kinetic energy (EKE) and a measure for $T'$ can be obtained from taking the square root of the eddy potential energy (EPE) \citep{VonStorch2012}. 
\cite{Gill1974a} noted that changes in stratification in the upper few hundred meters can lead to large changes in stability properties; decreasing stratification leads to greater conversion from potential energy to kinetic energy i.e. increased baroclinic instability \citep{Legg2001}. This is exactly what we see happening when the diabatic forcing in the northern sponge layer decreases.

\begin{figure}
\noindent \includegraphics[width=\textwidth]{../../Figures/EKE1.pdf} 
\caption{EKE $\frac{1}{2}(u^2 + v^2)$ for relaxation time scales of 3, 300, 3000 days and no relaxation. Isotherms in multiples of 1$^{\circ}$C are overlaid as solid black contours. The light grey contour indicates the mixed layer depth. Above the surface heat forcing is displayed. }
\label{fig:EKE_flat}
\end{figure}

Figure \ref{fig:EKE_flat} shows the zonally averaged EKE, with isotherms overlaid in black, and a thick grey line to indicate the mixed layer depth. Both the mixed layer depth and EKE increase, leading to large increases in diabatic eddy heat transport when the relaxation time scale in the sponge layer is increased. In the 3-day $\tau _R $ mixed layer depth is almost similar to the 300 day relaxation timescale layer depth, but once an internal boundary layer starts forming in the 3000-day and closed-basin runs, the mixed layer deepens and EKE increases. A redistribution in EKE occurs, decreasing in the north of the domain and increasing in the south and at depth, a pattern that we also observe in the diabatic eddy heat flux divergence in Fig.~\ref{fig:dhd}. As relaxation time scales increase to orders of tens of years and in the closed northern boundary case, a sharp increase in EKE with decreasing stratification occurs, in agreement with \cite{Gill1974a}. The change in mean kinetic energy between the runs is an order of magnitude smaller  and the increase in EPE in \fref{fig:EPE_flat} is qualitatively similar to the change in EKE.  
\begin{figure}
\noindent \includegraphics[width=\textwidth]{../../Figures/EPE1.pdf} 
\caption{EPE $\displaystyle{= g' T ^{\prime 2} \sfrac{d T^{*} }{dz}}$ for relaxation time scales of 3 days and no relaxation. Isotherms in multiples of 1$^{\circ}$C are overlaid as solid black contours. Above the surface heat forcing is displayed. }
\label{fig:EPE_flat}
\end{figure}


When considering kinetic energy we see a very clear trend of increasing KE and EKE with increasing relaxation time scale. Again this is almost a step wise response increasing drastically between 300 days and 3000 days. \fref{fig:KE_flat} shows a) the average increase in EKE corresponding to b) the increasing domain integrated KE. The increase in Kinetic energy is almost entirely confined to the EKE component.  

\begin{figure}[H]
\center
\subfloat[EKE][EKE]{\includegraphics[width=0.5\textwidth]{../../Figures/DEKE1.pdf}}
\subfloat[KE][KE]{\includegraphics[width=0.5\textwidth]{../../Figures/DKE1.pdf}}
\caption{Change in kinetic energy}
\label{fig:KE_flat}
\end{figure}
%%
%%%%%%%%%%%%%%%%%%%%%%%%%%%%%%%%%%%%%%%%%%%%%%%%%%
%%                Discussion                    %%
%%%%%%%%%%%%%%%%%%%%%%%%%%%%%%%%%%%%%%%%%%%%%%%%%%
%%
\section{Conclusions}

The diabatic eddy heat flux divergence strongly responds to changes in the northern boundary condition, becoming larger (increasing by 250\% in amplitude) when the stratification at the northern boundary is better able to freely evolve and is less constrained by the circulation and diabatic processes in the sponge layer. The resulting changes in diabatic forcing lead to a dramatic increase in surface mixed layer depth, which leads to enhanced baroclinic instability and larger EKE and EPE. This result is qualitatively robust to the surface boundary condition, but when surface restoring is applied instead of a flux formulation, a much smaller increase in diabatic fluxes occurs and the EKE is more constrained. Nevertheless, the role of diabatic eddy fluxes in the vertically integrated heat budget is equally large for both surface boundary conditions. It appears that the diabatic eddy heat flux divergence is sensitive to both the strength of the surface heat flux and the diabatic forcing in the sponge layer, but also to the surface boundary condition for temperature (buoyancy).

When the northern boundary is closed there can be no interior residual overturning circulation (where $\Psi _{iso} = \Phi _{res}$), although we see here $\Psi _{iso}$ reduces to near zero that is not necessarily a requirement. In the absence of diabatic forcing elsewhere the residual circulation will be confined to the surface mixed layer determined by surface heat fluxes and diabatic fluxes. When diabatic forcing becomes stronger in the sponge layer, the upper cell in the SO gains amplitude and becomes comparable to the observed SO ROC (and AMOC). In this regime the diabatic eddy heat flux divergence is always of first order importance, counteracting the heat transport by the ROC, leaving the surface forcing as a smaller residual. We did not find any regime where diabatic eddy fluxes can be neglected. 

The vertical integral of the diabatic eddy heat flux convergence decreases by an order of magnitude for weaker diabatic forcing in the sponge layer. Using fixed fluxes these changes are reconciled by the establishment of a strong internal boundary layer with a large vertical temperature gradient. 
As a result, both heat transport divergence by a weak ROC confined to strong northward deepening mixed layer, and the diabatic eddy heat flux divergence appear as dipoles of opposing signs. For the diabatic eddy fluxes heat convergence in the upper half of the internal boundary layer and heat divergence in the lower half occurs, while the heat transport divergence associated with the ROC in the mixed layer shows the opposite.
Without diabatic forcing in a northern boundary sponge layer diabatic eddies cancel the effective surface buoyancy forcing, while the heat transport divergence by the ROC integrates to zero in the vertical. As a result, below the surface mixed layer the SO ROC completely collapses, because  the connection to an adiabatic pole-to-pole circulation ceases to exist. 

The upper cell SO ROC collapses when diabatic forcing in the northern sponge-layer is absent. 
These results underscore the interhemispheric link between the SO ROC and the northern hemisphere AMOC. Such links were previously demonstrated in \citet{Gnanadesikan2000} and \citet{Wolfe2011}. Here, by altering the northern boundary condition we showed how the SO ROC adjusts to changes in stratification at the northern end of the SO. Although some gradual changes can be seen with increasing relaxation timescale, this is more a stepped response between turning on or off the sponge layer when using relaxation timescale as the variable.  

It should be stressed, however, that the representation of the far-field forcing, i.e. NADW formation, by a sponge layer with a prescribed e-folding stratification is crude and should be tested against other ways of closing the SO ROC. Nevertheless, the absence of far-field forcing implies a disconnect between the SO ROC and the AMOC, and our results imply that in this case the upper cell of the SO ROC cannot be maintained. The large changes in out diapycnal fluxes indicate that diabatic eddy heat fluxes could play a crucial role in the adjustment process to such changes highlights the need for a carefully designed diabatic eddy representation in the surface mixed layer of the ocean, which should also depend on the atmospheric state and forcing. Our results also imply that although there is a significant step between the reduced diabatic forcing in sponge layers with relaxation time scale of under a year and multiple years there are already changes in the dynamics of a channel model before we see the full destruction of the SO ROC.






% Chapter 3
\chapter[Surface Conditions]{Altering the surface boundary condition}
\input{surfacerestoring.tex}
% Chapter 4
\input{Preliminary_Experiments.tex}
\newpage
\input{Part2Outline.tex}
% Chapter 5
\chapter[Other Considerations]{Other considerations}
\label{chap:5}
%%%%%%%%%%%%%%%%%%%%%%%%%%%%%%%%%%%%
%\textcolor{red}{\textbf{What about Heat fluxes and the ACC? To be covered:}}
%\begin{itemize}
%\item[]\textcolor{red}{Heat Fluxes}
%\item[]\textcolor{red}{ACC baroclinic / tropic}
%\item[]\textcolor{red}{Saturation vs compensations}
%\end{itemize}

One of the major considerations we have neglected thus far is the influence on topography and the response of the ACC. Although beyond the scope of this thesis to fully include the role of topography on moderating the SO ROC response to altered boundary conditions in a channel model, we can include additional runs to validate and questions our conclusions drawn thus far. In this chapter we show the results for experiments including topography and discuss how this differs from flat bottom runs in the previous chapters.  

\section{Including topography}
%4) How does topography affect the aforementioned processes, in particular by reducing the scale depth of the ROC and diabatic eddies and reducing meridional isopycnal slopes in the interior?
%5). Topography comes last and should be used to illustrate how it affects diabatic eddies, heat flux divergence and ACC strength, and surface ROC in restoring runs and put them further into context..
% This means running addition surface resorting runs from spun up fixed flux runs.
In the real ocean, the wind stress applied to the surface of the ocean is balanced by bottom stresses and topography plays a leading order role in setting the spatial variability of the eddy field \citep{Thompson2008} as well as the efficiency of eddy dynamics \cite{Abernathey2014}. The southern ocean has areas of rough topography notably the Kerguelen Plateau and the Macquarie Ridge play large roles in setting steering the ACC and enhancing EKE and eddy mixing. As topography plays an important role in southern ocean dynamics it is important to consider what qualitative impacts it may have in our idealised set up and results. 

\subsection{Set up}
As the Southern Ocean is a region of important topographical interactions \citep{Abernathey2014} we checked whether these results also hold in the presence of topography. To investigate this we devised a further set of experiments. The model is once again based on the set up described in \fref{sec:Setup_stand} and we give a brief overview of the exact set up used here. To this end, we increased the domain size to 4000 km by 2000 km allowing for standing eddies. We also  increased the  domain depth to 4000 m, increasing the number of vertical grid boxes to 40. We then included a continental slope and 2 complex ridges as shown in Fig. \ref{fig:Topo}, similar to that used in previous studies \citep{hogg2010}. We use partial cells in the MITgcm to account for the topography.
\begin{figure}[H]
\center
\noindent \includegraphics[width=0.7\textwidth]{../../Figures/topo.pdf} 
\caption{Topography included in non-flat bottom runs.}
\label{fig:Topo}
\end{figure}
To reduce spurious diapycnal mixing advection scheme 7 was chosen \citep{hill2012, Ilıcak2012} and turbulent kinetic energy (TKE) scheme \citep{Gaspar1990,Madec1998}  was employed to parametrize sub grid mixing. Key model parameters are outlined in Table~\ref{tab:setup2}.

\begin{table}[H]
\caption{Model Setup for fixed-flux runs with topography.}
\label{tab:setup2}
\begin{center}
\begin{tabular}{|c|c|c|}
\hline \hline
\textbf{Symbol} & \textbf{Description}  & \textbf{Value} \\ 
\hline 
L$_x$, L$_y$, H & Domain & 4000 km, 2000 km, 4000 m \\ 
\hline 
L$_{sponge}$ & Length scale of sponge layer & 100 km \\ 
\hline 
Q$_0$ & Surface heat flux magnitude maximum & 0-10 W m$^{-2}$ \\ 
\hline 
$\tau _0$ & Max surface wind stress & 0.2 N m$^{-2}$ \\ 
\hline 
dx, dy & Horizontal grid spacing & 5 km  \\ 
\hline 
dz & Vertical grid spacing & 5-266 m \\ 
\hline 
Adv Scheme & 7$^{\text{th}}$ order centred & 7 \\ 
\hline 
 $\tau_{R}$ & Sponge relaxation time scale & 3-day and  $\infty$  \\
\hline 
\end{tabular}
\end{center}
\end{table}

The surface heat and wind forcing remain unchanged from the fixed flux runs. To test the effect of topography we ran 2 experiments for the 2 extreme relaxation timescales: 3 days and infinite (closed northern boundary). These simulations were run for 400 and 1000 years respectively.

\subsection{Overturning}
\begin{figure}[H]
\noindent \includegraphics[width=\textwidth]{../../Figures/instT.pdf} 
\caption{Instantaneous temperature field for a) and b) Topography runs. c) and d) flat bottom runs.}
\label{fig:instT}
\end{figure}

First we plot at the 3D instantaneous temperature field for both the flat runs and runs with topography (shown in grey) to highlight the main differences we see increased variability in in the topography runs \fref{fig:instT}.a-b and flatter isotherms. We still see a boundary layer form in the closed run.
As with all runs we check the Eulerian mean overturning looks sensible and consistent with all other runs. \fref{fig:MOC_full} shows the Eulerian MOC with topography is around 6 times larger (15 Sv) than the flat bottom runs (2.25 Sv). This increase is as expected with the 4 times increase in channel length alongside a 25\% increase in depth.
\begin{figure}[H]
\noindent \includegraphics[width=\textwidth]{../../Figures/MOC_flatvtopo.pdf} 
\caption{The Eulerian mean streamfunction ($\overline{\Psi}$) for Flat bottom runs a) and c) and for Topography runs b) and d).} 
\label{fig:MOC_full}
\end{figure}
%Dave said
%What does your bathymetry look like? Does it happen to have ridge, bumps, or something located at the latitude of the 3 lobes that you’re seeing in the EMOC? Its strikes me as possibly an external mode, or whatever the right term is, stemming from interactions of the mean flow with bathymetry. Something just occurred to me - how high off the bottom does your bathymetry come? Is it higher than the bottom of the lobes?
We also see a different structure in the runs with topography \fref{fig:MOC_full}, which is seen as a loped structure in b and d. This pattern is seen to be unchanged between averaging periods of 20 years to 100 years. Our ridges have some shallow peaks extending into the upper 1000m (as does the real southern ocean topography \citep{Smith and Sandwell, 1997}) of the ocean and the lobes may be due to the locations of the peaks as the overturning circulation must be above the topography \citep{viebahn2012}. As the "Deacon cell" \citep{Doos1994} is a fictitious cell we will not focus on the structure of the Eulerian mean overturning but will consider interaction with the mean flow in other analysis.
\begin{figure}[H]
\noindent \includegraphics[width=\textwidth]{../../Figures/ROC_flatvtopoT.pdf} 
\caption{The isothermal stream function $\Psi_{res}(y,\theta)$ for relaxation time scales of 3 and infinite days for flat bottom a) and c) runs and topography runs b) and d).}
\label{fig:RemapT_full}
\end{figure}


As with the previous chapters we calculate an isothermal stream function to show the ROC. However for the topography runs we calculated the stream function over 82 discrete potential temperature layers, in order to account for areas with little temperature variation at depth. The isothermal stream function is shown in \fref{fig:RemapT_full}, showing a similar 3 cell ROC pattern aligned with isotherms.
Fig. \ref{fig:Remap_full}.b shows that the SO ROC remapped into depth in these runs closely matches the 3 cell circulation in the flat bottom runs which collapses \footnote{The full spin up of the closed boundary run would take 3000 years, which is computationally too expensive. However, we know from the flat bottom experiments that the intense surface wind driven cell will disappear after 1000 years.}. The overturning strength is a factor of 5 stronger as expected with the longer channel. We discuss here the qualitative differences between the equivalent runs with flat bottoms. in \fref{fig:Remap_full}.b demonstrating that the fundamental ideas of how the SO ROC responds to stratification changes in the north of the domain still hold in models including topography. It should be noted that the lowermost cell of the SO ROC is still unrealistic in this idealised set up, as it is driven by diabatic processes not included in the idealised model. Its disappearance in these runs even in the presence of topography is not further discussed.

\begin{figure}[H]
\noindent \includegraphics[width=\textwidth]{../../Figures/ROC_flatvtopo.pdf} 
\caption{The isothermal stream function $\Psi_{res}(y,\theta)$ remapped onto depth coordinates, to give $\Psi_{res}(y,z)$ for relaxation time scales of 3 and infinite days for flat bottom a) and c) runs and topography runs b) and d). Isotherms in intervals of 1$^{\circ}$C are overlaid as solid black contours. Above the surface heat forcing is displayed.}
\label{fig:Remap_full}
\end{figure}

As well as the 3 cell SO ROC an its subsequent collapse we see that the same pattern of stratification change between the open and closed northern boundary case in the presence of topography. The sharp internal boundary layer that forms in the closed scenario appears much less sharp in the runs with the topography runs. We also see a shallowing of the mixed layer depth in the presence of topography for both the open and closed scenarios. Shoaling from 1200 m at the northern boundary in \fref{fig:Remap_full}.c to 500 m in \fref{fig:Remap_full}.d. 



\subsection{Dynamics}

Topography has a large influence on eddy dynamics with standing eddies playing a large role in eddy heat fluxes. However if we look at the depth profile of the eddy heat fluxes in \fref{fig:wptpfull} we see a reduction in magnitude between topography and no topography runs, but the same general pattern between open and closed northern boundaries. Notably we see a large deviation in the total heat flux between flat and topography runs with topography having larger downwards heat fluxes \fref{fig:wptpfull}.b and d.
\begin{figure}[H]
\noindent \includegraphics[width=\textwidth]{../../Figures/wptpfull.pdf} 
\caption{Vertical eddy heat flux (PW) for flat bottom runs a) and c) and topography runs b) and d). Mean vertical heat flux ($\overline{w}\overline{T}$) is show in red, Eddy($\overline{w'T'}$)  in blue and Total ($\overline{wT}$). }
\label{fig:wptpfull}
\end{figure}
We can see this in the diabatic eddy heat flux divergence terms. The vertically integrated values shown in \fref{fig:DExy} show the convergence in uniform bands in the \fref{fig:DExy}.a and c however we see locally very enhance divergence and convergence around topography \fref{fig:DExy}.b and d at over 100 times larger magnitude but confined to very short zonal areas except for over the continental shelf showing that although the net effect cancelling out the SO ROC still holds this is achieved around topography in strong dipole patterns.
\begin{figure}[H]
\noindent \includegraphics[width=\textwidth]{../../Figures/DEfullxy.pdf} 
\caption{Vertical integrated diapycnal eddy heat flux divergence,for flat bottom runs a) and c) and topography runs b) and d).}
\label{fig:DExy}
\end{figure}

\subsubsection*{Energetics}

\begin{figure}[H]
\noindent \includegraphics[width=\textwidth]{../../Figures/EKE_flatvtopo.pdf} 
\caption{EKE $\frac{1}{2}(u^2 + v^2)$ sections for relaxation time scales of 3 and infinite days for flat bottom a) and c) runs and topography runs b) and d). Isotherms in intervals of 1$^{\circ}$C are overlaid as solid black contours.}
\label{fig:EKE_full}
\end{figure}
In presence of topography we see an similar magnitudes of EKE maxima (within 10\%) to the flat bottom scenario but a much higher domain average (greater than twice flat bottom domain average).
\begin{figure}[H]
\noindent \includegraphics[width=\textwidth]{../../Figures/EKE_flatvtopo_xy.pdf} 
\caption{Vertically integrated EKE $\frac{1}{2}(u^2 + v^2)$ for relaxation time scales of 3 and infinite days for flat bottom a) and c) runs and topography runs b) and d).}
\label{fig:EKExy}
\end{figure}
As zonal means are not very appropriate in the presence of topography it is best to look at the vertically integrated EKE. As  expected from satellite altimeter data EKE in the southern ocean is patchy with large enhancement downstream of topography \citep{Thompson2008}.


\section{ACC}
%3) What is the role of the ACC-strength and mean isopycnal slope in the ROC and diabatic eddy response and how does it change in response to changes in buoyancy forcing?
%3. we should try to explain the response in terms of what determines the strength of the diabatic heat flux divergence (some of the deleted scaling of part 1 might come back). We should look at ACC-strength and isopycnal slopes as well.
Part of the unique dynamics as discussed in the introduction is the \gls{ACC} like the the SO ROC the ACC is driven by wind and buoyancy forcing. In our channel runs we have seen large changes in EKE and stratification and although wind forcing is kept constant we are likely to see a response in the ACC.
The ACC is set by a number of factors: Wind Stress ($\tau$), Surface buoyancy forcing and remote forcing altering stratification the most basic way of calculating transport is:
\begin{equation}
T=\iint u \mathrm{d}z\mathrm{d}y
\label{eq:transport}
\end{equation}
If you substitute thermal wind into \fref{eq:transport}, where thermal wind is given as:
\begin{equation}
u_z=\frac{u_z}{f\rho_0}\rho_y
\label{eq:thermalwind}
\end{equation}
Transport can be written:
\begin{equation}
T=\frac{g}{f\rho_0}\iint \rho_N -\rho_S \, \mathrm{d}z^2
\label{eq:transport2}
\end{equation}

Showing the role of stratification in setting ACC strength \citep{Hogg2010}. 


\subsection{The ACC response to altered northern boundary conditions}

We examine here what happens to the ACC when you alter the northern boundary condition with fixed surface fluxes. To illustrate this we have looked at the fixed surface heat flux and surface restoring runs to see how the closing of the northern boundary effects the ACC as well as the surface boundary condition. 

\begin{figure}[H]
\noindent \includegraphics[width=\textwidth]{../../Figures/Uflat.pdf} 
\caption{Zonal mean zonal velocity $\overline{u}$ m/s for a) Surface restoring with strong sponge layer, b) fixed heat fluxes and strong sponge layer, c) Surface restoring with Closed northern boundary and d) fixed heat fluxes with a closed northern boundary.}
\label{fig:Uflat}
\end{figure}

From varying the relaxation timescale at the northern boundary the ACC varies drastically, particularly the baroclinic component. \fref{fig:Uflat} show how little surface boundary conditions affect the zonal velocities. A very slight increase in velocity is noted with the marginally steeper isotherms in the fixed surface flux runs \fref{fig:Uflat}.b-d. For both surface heat forcing conditions we see the large effect that closing the northern boundary has on the zonal flow. When the northern boundary is closed a very large temperature (density) difference is achieved from the North of the ACC to the South with the sharp boundary layer leading the same difference across the ACC as the entire top to bottom temperature difference. 

\begin{figure}[H]
\noindent \includegraphics[width=\textwidth]{../../Figures/baroflat.pdf} 
\caption{Zonal mean barotropic component of zonal velocity for a) Surface restoring with strong sponge layer, b) fixed heat fluxes and strong sponge layer, c) Surface restoring with Closed northern boundary and d) fixed heat fluxes with a closed northern boundary.}
\label{fig:barotroflat}
\end{figure}

Breaking down into barotropic (\fref{fig:barotroflat}) and baroclinic (\fref{fig:baroclinflat}) components we see that the surface heat forcing has little effect on the transport with a 3 day relaxation timescale. The major changes between open and closed northern boundary occurs in the baroclinic component although a slight increase in the barotropic component is noted. Interestingly the barotropic transport is more noticeably increased when the northern boundary is closed with fixed surface heat fluxes, this could suggest that the choice of surface heat forcing has an effect on degree of eddy saturation which corroborates the findings of \citet{Zhai2014} who found shorter time scales of surface restoring increased eddy saturation. 

\begin{figure}[H]
\noindent \includegraphics[width=\textwidth]{../../Figures/baroclinflat.pdf} 
\caption{Zonal mean baroclinic component of zonal velocity for a) Surface restoring with strong sponge layer, b) fixed heat fluxes and strong sponge layer, c) Surface restoring with Closed northern boundary and d) fixed heat fluxes with a closed northern boundary.}
\label{fig:baroclinflat}
\end{figure}

When the northern boundary is closed we see a large increase in baroclinic transport, this increase is mostly seen in the surface mixed layer but we see this increase extend to depth.





% Chapter 5
% Discussion
\chapter{Discussion}
\input{discussion.tex}

%---------------------------------------------------------------------------------------------
\backmatter
\glsaddall{type=symbolslist} 

\printacronyms
\printglossary[type=symbolslist,style=symbunitlong]   % list of symbols
\printglossary[type=main]                     % main glossary
\bibliographystyle{agu}
\singlespacing
\bibliography{../SOMOC2}
\begin{appendices}
% Force nice layout in TOC and use letters (\Alph)
\renewcommand{\thesection}{\Alph{section}}
\section{Advective assumptions on model grid}
\label{sec:appendix1}




Here we demonstrate, by using a one dimensional example of Eq.~\ref{eq:adv}, that this equation is not valid on the model grid. In our model runs, we used the MITgcm advection scheme 7 \citep{Daru2004} with modifications from Adcroft and Campin, (MITgcm support mailing list communications 2010). This is a one step monotonicity-preserving 7th order scheme. Referring to \citet{Adcroft2004} and MITgcm model code \citep{marshall1997}, for a 1-D advection problem in the y-direction for a tracer Q the discrete fluxes become:

\begin{eqnarray}\nonumber
\nonumber \frac{\partial Q}{\partial t} + v\frac{\partial Q}{\partial y} +  = \\
\nonumber &&  \frac{1}{\Delta t} \left( Q_j ^{n+1} - Q_j ^{n} \right) + \frac{v}{\Delta y} \left(  Q_j ^n + \frac{1}{2} (1-C)(Q_{j+1} ^{n} - Q_j ^{n})  - \right. \nonumber  \\ 
&& \left. \frac{1}{6}  (1-C^2 )(Q_{j+1} ^{n} - 2Q_j ^{n} + Q_{j-1} ^{n} ) - Q_{j-1} ^n - \right. \\ 
\nonumber && \left. \frac{1}{2} (1-C)(Q_{j} ^{n} - Q_{j-1} ^{n})  + \right. \\  \nonumber  
&& \left. \frac{1}{6}  (1-C^2 )(Q_{j} ^{n} - 2Q_{-j} ^{n} + Q_{j-2} ^{n} )
\right) + O^4
\label{EQ:osmp7}
\end{eqnarray}

where C is the Courant number $\displaystyle{\left( \frac{u \Delta t }{\Delta x} \right) }$. Monotonicity-preserving flux limiters not described here are also applied. This 1-D treatment can be generalized to 3-D by removing the local divergence flow separately in each dimension. Using the identity:

\begin{equation}
v \frac{\partial Q}{\partial y} =\frac{\partial v Q}{\partial y} - Q \frac{\partial v}{\partial y},
\end{equation}

and applying Eq.~\ref{EQ:osmp7} on a C-grid (\fref{fig:gridfig}) we can demonstrate Eq.~\ref{eq:adv} does not hold on the model grid.

\begin{eqnarray}\nonumber
\nonumber \frac{\partial Q}{\partial t} + v\frac{\partial Q}{\partial y} +  = \\
\nonumber &&
\left. \frac{1}{\Delta t}
\left( Q_j ^{n+1} - Q_j ^{n} \right) 
+ \frac{\left( \frac{v_{j-\frac{1}{2}}+v_{j+\frac{1}{2}}}{2}\right)}{\Delta y} 
( 
Q_j ^n + \right. \\ \nonumber
&& \left.
\frac{1}{2} (1-C)(
Q_{j+1} ^{n} 
Q_j ^{n})  - \frac{1}{6}  (1-C^2 )
Q_{j+1} ^{n} -\right. \\ \nonumber
&& \left.
 2Q_j ^{n}
+ Q_{j-1} ^{n} )-Q_{j-1} ^n -
\frac{1}{2} (1-C)(
 Q_{j} ^{n} -
\right. \\ \nonumber 
&& \left.
Q_{j-1} ^{n})  
+   \frac{1}{6}  (1-C^2 )(
Q_{j} ^{n} - 2
Q_{j-1} ^{n} + 
\right. \\ 
&& \left. 
 Q_{j-2} ^{n} )
 + O^4 \right.
\label{EQ:unabT}
\end{eqnarray}

whereas, when velocity is taken back inside the divergence term:

\begin{eqnarray}\nonumber
\nonumber \frac{\partial vQ}{\partial t} + \frac{\partial vQ}{\partial y} +  = \\
\nonumber &&
\left. \frac{1}{\Delta t}
\left( \left(v_{j-\frac{1}{2}}\left(\frac{Q_j+Q_{j-1}}{2}\right)\right) ^{n+1} - \left(v_{j-\frac{1}{2}}(\frac{Q_j+Q_{j-1}}{2})\right) ^{n} \right) 
+ 
\right. \\ \nonumber
 &&\left.
\frac{1}{\Delta y} 
( \left( \frac{Q_j+Q_{j-1}}{2} \right)
v_{j-\frac{1}{2}} ^n + \right. \\ \nonumber
&& \left.
\frac{1}{2} (1-C)(
\left(\frac{Q_j+Q_{j+1}}{2}\right)
v_{j+\frac{1}{2}} ^{n} - \right. \\ \nonumber
&& \left.
\left(  \frac{Q_j+Q_{j-1}}{2} \right)
v_{j-\frac{1}{2}} ^{n})  - \frac{1}{6}  (1-C^2 )
\left( \frac{Q_j+Q_{j+1}}{2} \right)
v_{j+\frac{1}{2}} ^{n} -\right. \\ \nonumber
&& \left.
 2\left( \frac{Q_j+Q_{j-1}}{2}\right)v_{j-\frac{1}{2}} ^{n}
+ \left( \frac{Q_{j-1}+Q_{j-2}}{2}\right)v_{j-\frac{3}{2}} ^{n} )-
\right. \\ \nonumber 
&& \left.
\left(\frac{Q_{j-1}+Q_{j-2}}{2}\right) v_{j-\frac{3}{2}} ^n -
\frac{1}{2} (1-C)(
\left( \frac{Q_j+Q_{j-1}}{2} \right) v_{j-\frac{1}{2}}  ^{n} -
\right. \\ \nonumber 
&& \left.
\left( \right)v_{j-\frac{3}{2}} ^{n})  
+   \frac{1}{6}  (1-C^2 )(
\left( \frac{Q_j+Q_{j-1}}{2}\right)v_{j-\frac{1}{2}} ^{n} - 2
\left( \frac{Q_{j-1}+Q_{j-2}}{2}\right)v_{j-\frac{3}{2}} ^{n} + 
\right. \\ 
&& \left. 
\left( \frac{Q_{j-2}+Q_{j-3}}{2}\right) v_{j-\frac{5}{2}} ^{n} )
 + O^4 \right.
\label{EQ:nabuT}
\end{eqnarray}

One can see by comparing Eq.~\ref{EQ:unabT} and Eq.~\ref{EQ:nabuT} that they are not equivalent. Terms of 2$^{\text{nd}}$ and higher order differ, indicating Eq.~\ref{eq:adv} can not be used when evaluating discretized numerical models.

\section{Eaxmple datafile}
\label{sec:appendix2}


To give a full overview of the parameters used in the standard model set up below is an example data input file with the run time model flags and in italics a brief explanation. 

{\setlength{\parindent}{0cm}
\& \textbf{PARM01}

\textbf{tRef=Reference Profile,} \textit{This the reference temperature used in things like linear EOS, set as the sponge profile} 

\# viscosity 

\textbf{viscAh=12.000000,} \textit{Laplacian viscosity coefficient ($m^2/s$) Set so that horizontal Reynolds number is ~0, this scales with resolution of one less than the order (so linearly here) - basically it is to ensure that the velocity is not too fast that a wave can pass through a cell in less than one time step with the scaling relation  $\displaystyle A_h < \frac{L^2}{4 \Delta t}$ (Courant-Freidrichs-Lewy Constraint on viscosity). \citet{griffies2000} notes that it actually scales to one less power.}
 
\textbf{viscAz=0.000300,}\textit{vertical eddy coefficient. default}
 
\textbf{viscA4=9.0E8,}\textit{Bi-harmonic viscosity coefficient ($m^4/s$) With the relation $\displaystyle A_4 <= \frac{L^4}{32 \Delta t}$}. Allows a less viscous yet numerically stable simulation than harmonic (laplacian) as eddy viscosities can be focuses on the dynamics at the grid scale as large motions would be 'resolved'. 
 
 
\textbf{viscA4GridMax=0.500}\textit{Maximum grid dependent biharmonic viscosity }

 
\textbf{viscAhGridMax=0.500}\textit{Maximum lateral grid dependent eddy viscosity}
 
\# Diffusion 
 
\textbf{diffKhT=0.000000,}\textit{Laplacian diffusion of heat laterally ($m^2/s$) the boundary condition on this operator is $\displaystyle \frac{\partial}{\partial x}=\frac{\partial}{\partial y}=0$ on all boundaries. Default.}
 
\textbf{diffKzT=0.00000,} \textit{Laplacian diffusion of salt vertically can be set implicitly by setting implicitDiffusion to ’.TRUE.’ Default.}
 
\textbf{diffKhS=0.000000,}\textit{Laplacian diffusion of salt laterally ($m^2/s$) same applies as for T. Default. }
 
\textbf{diffKzS=0.00000,}\textit{Laplacian diffusion of salt vertically ($m^2/s$) same applies as for S.  Default.}

\# Advection Scheme

\textbf{tempAdvScheme=7,}\textit{The seventh-order one-step method with monotonicit-preserving limiter (OS7MP) \citep{Daru2004} and minimises numerical diffusion \citep{Ilıcak2012,shakespeare2012}}

\# Timestepping

\textbf{StaggerTimeStep=.TRUE.,}\textit{The stagger baroclinic time stepping rather than synchronous time stepping. The thermodynamics solver is delayed from half a time step, allowing the use of the most recent velocities to compute the advection terms}

\textbf{saltStepping=.FALSE.,}\textit{Salinity equation off.}

\# equation of state

\textbf{eosType='LINEAR'},\textit{Sets linear EOS, buoyancyRelation is automatically set to OCEANIC. For the nonlinear case, you need to generate a file of polynomial coeffcients called
POLY3.COEFFS. To do this, use the program utils/knudsen2/knudsen2.f instead of tAlpha and sBeta}

\textbf{tAlpha=2.0E-4,}\textit{thermal expansion coefficient in $K^{-1}$ for EOS. Default.}

\textbf{sBeta =0.,}\textit{salt is passive here (ppt$^{-1}$)}
 
\# Boundary conditions

\textbf{no\_ slip\_ sides=.TRUE.,} \textit{Free slip or no slip. Free slip give zero stress on boundaries (more convenient to code). No slip defines the normal gradient of a tangential flow such that the flow is zero on the boundary by adding an additional source term in cells next to the boundary}

\textbf{no\_ slip\_ bottom=.TRUE.,}

\textbf{bottomDragLinear=1.1E-03,}\textit{linear bottom-drag coefficient ( m/s )}

\textbf{ bottomDragQuadratic=0.000000E+00,}\textit{quadratic bottom-drag coeff. Default.}

\# physical parameters

\textbf{f0=-1.E-4,}\textit{Coriolis parameter ($s^{-1}$), $\displaystyle f=2\omega sin \phi$ Here this is set to the negative reference value and as beta is non zero f0 is the value of f at the southern edge of the domain.}

\textbf{beta=1E-11,}\textit{$\displaystyle \frac{\partial f}{\partial y}$ ($m^{-1}s^{-1}$)}

\textbf{gravity=9.81,}

\# exact volume conservation

\textbf{exactConserv=.TRUE.,}

\# C-V scheme for Coriolis term

\textbf{useCDscheme=.FALSE.,}

\# partial cells for smooth topography

\textbf{HfacMin=0.05,} \textit{ Limit partial cells to 50m thickness}

\# file I/O parameters

\textbf{readBinaryPrec=64,}\textit{double precision} 

\textbf{ useSingleCpuIO=.TRUE.,}

\textbf{ debugLevel=1,}

\# Default surface conditions

\textbf{rigidLid=.FALSE.,}

\textbf{implicitFreeSurface=.TRUE.,}\textit{these lines suppress the rigid lid formulation of the surface pressure inverter and activate the implicit
free surface form of the pressure inverter.}

\textbf{\&}

\textbf{\& PARM02}

\textbf{cg2dMaxIters=500,}\textit{Upper limit on 2d con. grad iterations, default is 150.}

\textbf{cg2dTargetResidual=1.E-9,} \textit{default is 1.E-7}

\textbf{cg3dMaxIters=40,}\textit{Sets the maximum number of iterations the three-dimensional, conjugate gradient solver}

\textbf{cg3dTargetResidual=1.E-9,}


\textbf{\&}


\textbf{\& PARM03}

\textbf{deltaT=450,}\textit{time steps scales with resolution}

\# 7.5 minute time step -> 69120 time steps/year

\textbf{nIter0=0,}\textit{Start at..}

\textbf{nTimeSteps=1382400,}\textit{20 years}

\textbf{cAdjFreq = -1.,}\textit{Convective adjustment interval (s), the frequency at which the adjustment algorithm is called to a non-zero value, a negative value sets to tracer time step}

\textbf{abEps=0.1,}\textit{ Adams-Bashforth stabilizing parameter, required as staggerTimeStep is set to TRUE}

\# Permanent restart/checkpoint file interval ( s )

\# Try 10 year interval (two per run)

\textbf{pChkptFreq=311040000.00,}\textit{Checkpoint file interval (s)}

\textbf{dumpFreq=0.0,} \textit{Model state write out interval (s)}
 
\textbf{monitorFreq=0.0,} \textit{Monitor output interval (s)}

\textbf{dumpInitAndLast=.TRUE.,}

\textbf{pickupStrictlyMatch=.FALSE.,}

\textbf{\&}

\textbf{\& PARM04}

\textbf{usingCartesianGrid=.TRUE.,}\textit{Uses cartesian co-ordinate system with uniform grid space via dXspacing or dYspacing or by vectors DelX or DelY in (m)}

\textbf{usingSphericalPolarGrid=.FALSE.,}

\textbf{delXfile='delX',}\textit{Can be set as a 1D array file or as gridlines * spacing i.e 300*5E3 300 grid lines at 5km spacing}

\textbf{delYfile='delY',}

\textbf{delZfile='delZ'}

\textbf{\&}

\textbf{\& PARM05}

\textbf{surfQfile='Qsurface'}

\textbf{bathyFile='topog'}

\textbf{zonalWindFile='Wind',}

\textbf{meridWindFile=,}

\textbf{hydrogThetaFile='T.init',}

\textbf{\&}
}
\end{appendices}
\end{document}
